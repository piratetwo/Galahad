\documentclass{galahad}

% set the package name

\newcommand{\package}{fdc}
\newcommand{\packagename}{FDC}
\newcommand{\fullpackagename}{\libraryname\_\-\packagename}

\begin{document}

\galheader

%%%%%%%%%%%%%%%%%%%%%% SUMMARY %%%%%%%%%%%%%%%%%%%%%%%%

\galsummary
Given an under-determined set of linear equations/constraints
$\bma_i^T \bmx = c_i^{}$,
$i = 1, \ldots, m$ involving $n \geq m$ unknowns $\bmx$, this package
{\bf determines whether the constraints are consistent, and if so how many
of the constraints are dependent}; a list of dependent constraints, that
is, those which may be removed without changing the solution set, will be found
and the remaining $\bma_i$ will be linearly independent.
Full advantage is taken of any zero coefficients in the vectors $\bma_i$.

%%%%%%%%%%%%%%%%%%%%%% attributes %%%%%%%%%%%%%%%%%%%%%%%%

\galattributes
\galversions{\tt  \fullpackagename\_single, \fullpackagename\_double}.
\galuses
{\tt GALAHAD\_\-CLOCK},
{\tt GALAHAD\_SY\-M\-BOLS}, {\tt GALAHAD\_\-STRING},
{\tt GALAHAD\_SMT}, {\tt GALAHAD\_\-ROOTS}, {\tt GALAHAD\_\-SLS},
{\tt GALAHAD\_\-ULS}, {\tt GALAHAD\_SPECFILE}, {\tt GALAHAD\-\_SPACE}.
\galdate August  2006.
\galorigin N. I. M. Gould, Rutherford Appleton Laboratory.
\gallanguage Fortran~95 + TR 15581 or Fortran~2003.
\galparallelism Some options may use OpenMP and its runtime library.

%%%%%%%%%%%%%%%%%%%%%% HOW TO USE %%%%%%%%%%%%%%%%%%%%%%%%

\galhowto

%\subsection{Calling sequences}

Access to the package requires a {\tt USE} statement such as

\medskip\noindent{\em Single precision version}

\hspace{8mm} {\tt USE \fullpackagename\_single}

\medskip\noindent{\em Double precision version}

\hspace{8mm} {\tt USE  \fullpackagename\_double}

\medskip

\noindent
If it is required to use both modules at the same time, the derived types
{\tt SMT\_type},
{\tt \packagename\_time\_type},
{\tt \packagename\_control\_type},
and
{\tt \packagename\_inform\_type}
(Section~\ref{galtypes})
and the subroutines
{\tt \packagename\_initialize},
{\tt \packagename\_\-find\_dependent},
{\tt \packagename\_terminate}
(Section~\ref{galarguments})
and
{\tt \packagename\_read\_specfile}
(Section~\ref{galfeatures})
must be renamed on one of the {\tt USE} statements.

%%%%%%%%%%%%%%%%%%%%%% long integers %%%%%%%%%%%%%%%%%%%%%%

\subsection{Integer kinds}\label{Integer kinds}
We use the term
long \integer\ to denote {\tt INTEGER\-(kind=long)}, where
{\tt long = selected\_int\_kind(18))}.

%%%%%%%%%%%%%%%%%%%%%% OpenMP usage %%%%%%%%%%%%%%%%%%%%%%%%

\subsection{OpenMP}
OpenMP may be used by the {\tt \fullpackagename} package to provide
parallelism for some solver options in shared memory environments.
See the documentation for the \galahad\ package {\tt SLS} for more details.
To run in parallel, OpenMP
must be enabled at compilation time by using the correct compiler flag
(usually some variant of {\tt -openmp}).
The number of threads may be controlled at runtime
by setting the environment variable {\tt OMP\_NUM\_THREADS}.

\noindent
The code may be compiled and run in serial mode.

%%%%%%%%%%%%%%%%%%%%%% derived types %%%%%%%%%%%%%%%%%%%%%%%%

\galtypes
Three derived data types are accessible from the package.

%%%%%%%%%%% control type %%%%%%%%%%%

\subsubsection{The derived data type for holding control
 parameters}\label{typecontrol}
The derived data type
{\tt \packagename\_control\_type}
is used to hold controlling data. Default values may be obtained by calling
{\tt \packagename\_initialize}
(see Section~\ref{subinit}),
while components may also be changed by calling
{\tt \fullpackagename\_read\-\_spec}
(see Section~\ref{readspec}).
The components of
{\tt \packagename\_control\_type}
are:

\begin{description}

\itt{error} is a scalar variable of type default \integer, that holds the
stream number for error messages. Printing of error messages in
{\tt \packagename\_find\_dependent} and {\tt \packagename\_terminate}
is suppressed if {\tt error} $\leq 0$.
The default is {\tt error = 6}.

\ittf{out} is a scalar variable of type default \integer, that holds the
stream number for informational messages. Printing of informational messages in
{\tt \packagename\_find\_dependent} is suppressed if {\tt out} $< 0$.
The default is {\tt out = 6}.

\itt{print\_level} is a scalar variable of type default \integer, that is used
to control the amount of informational output which is required. No
informational output will occur if {\tt print\_level} $\leq 0$. If
{\tt print\_level} $= 1$, basic statistics of the performance of the
package will be produced. If {\tt print\_level} $\geq 2$, this output will be
increased to provide details such as the size of each neglected pivot.
The default is {\tt print\_level = 0}.

%\itt{indmin} is a scalar variable of type default \integer, that specifies
%an initial estimate as to the amount of integer workspace required by
%the factorization package {\tt SLS}.
%The default is {\tt indmin = 1000}.

%\itt{valmin} is a scalar variable of type default \integer, that specifies
%an initial estimate as to the amount of real workspace required by
%the factorization package {\tt SLS}.
%The default is {\tt valmin = 1000}.

\itt{max\_infeas} is a scalar variable of type default \realdp, that holds the
largest permitted infeasibility for a dependent constraint. Specifically, if
$\bmx$ satisfies $\bma_i^T \bmx = c_i^{}$ for the constraints deemed to be
linearly independent, it is required that $|\bma_i^T \bmx - c_i^{}| \leq$
{\tt max\_infeas} for those classified as dependent.
The default is {\tt max\_infeas =} $u^{1/3}$,
where $u$ is {\tt EPSILON(1.0)} ({\tt EPSILON(1.0D0)} in
{\tt \fullpackagename\_double}).

\itt{pivot\_tol} is a scalar variable of type default
\realdp, that holds the
relative pivot  tolerance used by the matrix factorization when
attempting to detect linearly dependent constraints.
%A value larger than  {\tt pivot\_tol} is appropriate.
See the documentation for the packages {\tt SLS} and {\tt ULS} for details.
The default is {\tt pivot\_tol = 0.5}.

%\itt{zero\_pivot} is a scalar variable of type default \realdp.
%Any pivots smaller than  {\tt zero\_pivot} in absolute value will be regarded
%to be zero when attempting to detect linearly dependent constraints.
%The default is {\tt zero\_pivot =} $u^{3/4}$,
%where $u$ is {\tt EPSILON(1.0)} ({\tt EPSILON(1.0D0)} in
%{\tt \fullpackagename\_double}).

\itt{use\_sls} is a scalar variable of type default
\logical, that should be set \true\ if the
\libraryname\ package {\tt SLS} is to be used
to detect linearly dependent constraints, or \false\ if
the \libraryname\ package {\tt ULS} is to be used instead.
The default is {\tt use\_sls = }\false.

\itt{scale} is a scalar variable of type default \logical,
that must be set \true\ if the rows of $\bmA$ are to be scaled to
have unit (infinity) norm and  \false\ otherwise.
The default is {\tt scale = .FALSE.}.

\itt{space\_critical} is a scalar variable of type default \logical,
that must be set \true\ if space is critical when allocating arrays
and  \false\ otherwise. The package may run faster if
{\tt space\_critical} is \false\ but at the possible expense of a larger
storage requirement. The default is {\tt space\_critical = .FALSE.}.

\itt{deallocate\_error\_fatal} is a scalar variable of type default \logical,
that must be set \true\ if the user wishes to terminate execution if
a deallocation  fails, and \false\ if an attempt to continue
will be made. The default is {\tt deallocate\_error\_fatal = .FALSE.}.

\itt{symmetric\_linear\_solver} is a scalar variable of type default \character\
and length 30, that specifies the external package to be used to
solve any symmetric linear system that might arise. Current possible
choices are {\tt 'sils'}, {\tt 'ma27'}, {\tt 'ma57'}, {\tt 'ma77'},
{\tt 'ma86'}, {\tt 'ma97'}, {\tt ssids}, {\tt 'pardiso'}
and {\tt 'wsmp'},
although only {\tt 'sils'} and, for OMP 4.0-compliant compilers,
{\tt 'ssids'} are installed by default.
See the documentation for the \galahad\ package {\tt SLS} for further details.
The default is {\tt symmetric\_linear\_solver = 'sils'},
but we recommend {\tt 'ma97'} if it available.

\itt{prefix} is a scalar variable of type default \character\
and length 30, that may be used to provide a user-selected
character string to preface every line of printed output.
Specifically, each line of output will be prefaced by the string
{\tt prefix(2:LEN(TRIM(prefix))-1)},
thus ignoreing the first and last non-null components of the
supplied string. If the user does not want to preface lines by such
a string, they may use the default {\tt prefix = ""}.

\itt{SLS\_control} is a scalar variable argument of type
{\tt SLS\_control\_type} that is used to pass control
options to external packages used to
factorize relevant symmetric matrices that arise.
See the documentation for the \galahad\ package {\tt SLS} for further details.
In particular, default values are as for {\tt SLS}, except that
{\tt SLS\_control\%rela\-tive\_pivot\_tolerance} is reset to
{\tt pivot\_tol}.

\itt{ULS\_control} is a scalar variable argument of type
{\tt ULS\_control\_type} that is used to pass control
options to external packages used to
factorize relevant unsymmetric matrices that arise.
See the documentation for the \galahad\ package {\tt ULS} for further details.
In particular, default values are as for {\tt ULS}, except that
{\tt ULS\_control\%rela\-tive\_pivot\_tolerance} is reset to
{\tt pivot\_tol}.

\end{description}

%%%%%%%%%%% time type %%%%%%%%%%%

\subsubsection{The derived data type for holding timing
 information}\label{typetime}
The derived data type
{\tt \packagename\_time\_type}
is used to hold elapsed CPU and system clock times for the various parts of
the calculation. The components of
{\tt \packagename\_time\_type}
are:
\begin{description}
\itt{total} is a scalar variable of type default \realdp, that gives
 the total CPU time spent in the package.

\itt{analyse} is a scalar variable of type default \realdp, that gives
 the CPU time spent analysing the required matrices prior to factorization.

\itt{factorize} is a scalar variable of type default \realdp, that gives
 the CPU time spent factorizing the required matrices.

\itt{clock\_total} is a scalar variable of type default \realdp, that gives
 the total elapsed system clock time spent in the package.

\itt{clock\_analyse} is a scalar variable of type default \realdp, that gives
 the elapsed system clock time spent analysing the required matrices prior to
factorization.

\itt{clock\_factorize} is a scalar variable of type default \realdp, that gives
 the elapsed system clock time spent factorizing the required matrices.

\end{description}

%%%%%%%%%%% info type %%%%%%%%%%%

\subsubsection{The derived data type for holding informational
 parameters}\label{typeinform}
The derived data type
{\tt \packagename\_inform\_type}
is used to hold parameters that give information about the progress and needs
of the algorithm. The components of
{\tt \packagename\_inform\_type}
are:

\begin{description}

\itt{status} is a scalar variable of type default \integer, that gives the
exit status of the algorithm.
%See Sections~\ref{galerrors} and \ref{galinfo}
See Section~\ref{galerrors}
for details.

\itt{alloc\_status} is a scalar variable of type default \integer, that gives
the status of the last attempted array allocation or deallocation.
This will be 0 if {\tt status = 0}.

\itt{bad\_alloc} is a scalar variable of type default \character\
and length 80, that  gives the name of the last internal array
for which there were allocation or deallocation errors.
This will be the null string if {\tt status = 0}.

\itt{factorization\_status} is a scalar variable of type default \integer, that
gives the return status from the matrix factorization.

\itt{factorization\_integer} is a scalar variable of type long
\integer, that gives the amount of integer storage used for the matrix
factorization.

\itt{factorization\_real} is a scalar variable of type long \integer,
that gives the amount of real storage used for the matrix factorization.

\itt{non\_negligible\_pivot} is a scalar variable of type default \realdp,
that holds the value of the smallest pivot larger than
{\tt control\%zero\_pivot}
when searching for dependent linear constraints. If
{\tt non\_negligible\_pivot} is close to  {\tt control\%zero\_pivot},
this may indicate that there are further dependent constraints, and
it may be worth increasing {\tt control\%zero\_pivot} above
{\tt non\_negligible\_pivot} and solving again.

\ittf{time} is a scalar variable of type {\tt \packagename\_time\_type}
whose components are used to hold elapsed CPU and system clock
times for the various parts of the calculation (see Section~\ref{typetime}).

\itt{SLS\_inform} is a scalar variable argument of type
{\tt SLS\_inform\_type} that is used to pass information
concerning the progress of the external packages used to
factorize relevant symmetric matrices that arise.
See the documentation for the \galahad\ package {\tt SLS} for further details.

\itt{ULS\_inform} is a scalar variable argument of type
{\tt ULS\_inform\_type} that is used to pass information
concerning the progress of the external packages used to
factorize relevant symmetric matrices that arise.
See the documentation for the \galahad\ package {\tt ULS} for further details.

\end{description}

%%%%%%%%%%%%%%%%%%%%%% argument lists %%%%%%%%%%%%%%%%%%%%%%%%

\galarguments
There are three procedures for user calls
(see Section~\ref{galfeatures} for further features):

\begin{enumerate}
\item The subroutine
      {\tt \packagename\_initialize}
      is used to set default values
      before attempting to identify dependent constraints.
\item The subroutine
      {\tt \packagename\_find\_dependent}
      is called to identify dependent constraints.
\item The subroutine
      {\tt \packagename\_terminate}
      is provided to allow the user to automatically deallocate
      workspace array components, previously allocated by
       {\tt \packagename\_find\_dependent}, after use.
\end{enumerate}
We use square brackets {\tt [ ]} to indicate \optional arguments.

%%%%%% initialization subroutine %%%%%%

\subsubsection{The initialization subroutine}\label{subinit}
 Default values are provided as follows:
\vspace*{1mm}

\hspace{8mm}
{\tt CALL \packagename\_initialize(  data, control, inform )}

\vspace*{1mm}
\begin{description}

\itt{data} is a scalar \intentinout\ argument of type
{\tt \packagename\_data\_type}
that need not be set on entry and whose components will be used as workspace.

\itt{control} is a scalar \intentout\ argument of type
{\tt \packagename\_control\_type}
(see Section~\ref{typecontrol}).
On exit, {\tt control} contains default values for the components as
described in Section~\ref{typecontrol}.
These values should only be changed after calling
{\tt \packagename\_initialize}.

\itt{inform} is a scalar \intentinout\ argument of type
{\tt \packagename\_inform\_type}
(see Section~\ref{typeinform}). A successful call to
{\tt \packagename\_find\_dependent}
is indicated when the  component {\tt status} has the value 0.
For other return values of {\tt status}, see Section~\ref{galerrors}.

\end{description}

%%%%%%%%% main solution subroutine %%%%%%

\subsubsection{The subroutine for finding dependent constraints}
Dependent constraints are identified as follows:

\vspace*{1mm}

\hspace{8mm}
{\tt CALL \packagename\_find\_dependent( n, m, A\_val, A\_col, A\_ptr, C,
  \&}
\vspace*{-1mm}

\hspace{52mm}
{\tt    n\_depen, C\_depen, data, control, inform )}

%\vspace*{1mm}

\begin{description}

\ittf{m} is a scalar \intentin\ argument of type default \integer,
that holds the number of constraints, $m$.
{\bf Restriction: } $m \geq 0$.

\ittf{n} is a scalar \intentin\ argument of type default \integer,
that holds the number of unknowns, $n$.
{\bf Restriction: } $n \geq 0$.

\itt{A\_val} is an \intentin\ rank-one array of type default \realdp, that holds
the values of the entries (that is those component whose values are nonzero)
of the matrix $\bmA$ whose rows are the vectors $\bma_i^T$, $i = 1, \ldots, m$.
The entries for row $i$ must directly precede those in row $i$, but the
order within each row is unimportant.

\itt{A\_col} is an \intentin\ rank-one array of type default \integer, that holds
the (column) indices of the entries of $\bmA$ corresponding to the
values input in {\tt A\_val}.

\itt{A\_ptr} is an \intentin\ rank-one array of dimension {\tt m+1} and type
default \integer, whose $i$-th entry holds the starting position of row $i$
of $\bmA$ for $i = 1, \ldots, m$. The $m+1$-st entry of {\tt A\_ptr}
must hold the total number of entries plus one.

\ittf{C} is an \intentin\ rank-one array of dimension {\tt m} and
type default \realdp, whose $i$-th component must be set to $c_i$ for
$i = 1, \ldots, m$.

\itt{n\_depen} is a scalar \intentout\ argument of type default \integer,
that gives the number of dependent constraints.

\itt{C\_depen} is a rank-one allocatable array of type default \integer,
that should either have been nullified or associated on entry,
but need not be set.
On exit, if {\tt n\_depen} $>$ {\tt 0}, it will have been allocated to be of
length {\tt n\_depen} and its components will be the indices of the dependent
constraints. It will not be allocated or set if {\tt n\_depen = 0}.

\itt{data} is a scalar \intentinout\ argument of type
{\tt \packagename\_data\_type}
that need not be set on entry and whose components will be used as workspace.

\itt{control} is a scalar \intentin\ argument of type
{\tt \packagename\_control\_type}
(see Section~\ref{typecontrol}). Default values may be assigned by calling
{\tt \packagename\_initialize} prior to the first call to
{\tt \packagename\_find\_dependent}.

\itt{inform} is a scalar \intentinout\ argument of type
{\tt \packagename\_inform\_type}
(see Section~\ref{typeinform}). A successful call to
{\tt \packagename\_find\_dependent}
is indicated when the  component {\tt status} has the value 0.
For other return values of {\tt status}, see Section~\ref{galerrors}.

\end{description}

%%%%%%% termination subroutine %%%%%%

\subsubsection{The  termination subroutine}
All previously allocated arrays are deallocated as follows:
\vspace*{1mm}

\hspace{8mm}
{\tt CALL \packagename\_terminate( data, control, inform[, C\_depen ])}

\vspace*{-1mm}
\begin{description}

\itt{data} is a scalar \intentinout\ argument of type
{\tt \packagename\_data\_type} whose array components will be deallocated
on exit.

\itt{control} is a scalar \intentin\ argument of type
{\tt \packagename\_control\_type}
exactly as for
{\tt \packagename\_find\_dependent}.

\itt{inform} is a scalar \intentinout\ argument of type
{\tt \packagename\_inform\_type}
exactly as for
{\tt \packagename\_find\_dependent}.
Only the components {\tt status}
{\tt alloc\_status} and {\tt bad\_alloc}
will be set on exit, and a
successful call to
{\tt \packagename\_terminate}
is indicated when this  component {\tt status} has the value 0.
For other return values of {\tt status}, see Section~\ref{galerrors}.

\itt{C\_depen} is an \optional\ rank-one allocatable array of type default
\integer, that will be deallocated on exit if \present.

\end{description}

%%%%%%%%%%%%%%%%%%%%%% Warning and error messages %%%%%%%%%%%%%%%%%%%%%%%%

\galerrors
A negative value of {\tt inform\%status} on exit from
{\tt \packagename\_find\_dependent}
or
{\tt \packagename\_terminate}
indicates that an error has occurred. No further calls should be made
until the error has been corrected. Possible values are:

\begin{description}

\itt{\galerrallocate.} An allocation error occurred.
A message indicating the offending
array is written on unit {\tt control\%error}, and the returned allocation
status and a string containing the name of the offending array
are held in {\tt inform\%alloc\_\-status}
and {\tt inform\%bad\_alloc} respectively.

\itt{\galerrdeallocate.} A deallocation error occurred.
A message indicating the offending
array is written on unit {\tt control\%error} and the returned allocation
status and a string containing the name of the offending array
are held in {\tt inform\%alloc\_\-status}
and {\tt inform\%bad\_alloc} respectively.
status is given by the value {\tt inform\%alloc\_status}.

\itt{\galerrrestrictions.} One of the restrictions
{\tt n} $\geq 0$ or {\tt m} $\geq  0$
    has been violated.

\itt{\galerrprimalinfeasible} The constraints appear to be inconsistent.

\itt{\galerranalysis.} The analysis phase of the factorization failed;
    the return status from the factorization
    package is given in the component {\tt inform\%fac\-t\-orization\_status}.

\itt{\galerrfactorization.} The factorization failed;
 the return status from the factorization
    package is given in the component {\tt inform\%fac\-t\-orization\_status}.

\end{description}

%%%%%%%%%%%%%%%%%%%%%% Further features %%%%%%%%%%%%%%%%%%%%%%%%

\galfeatures
\noindent In this section, we describe an alternative means of setting
control parameters, that is components of the variable {\tt control} of type
{\tt \packagename\_control\_type}
(see Section~\ref{typecontrol}),
by reading an appropriate data specification file using the
subroutine {\tt \packagename\_read\_specfile}. This facility
is useful as it allows a user to change  {\tt \packagename} control parameters
without editing and recompiling programs that call {\tt \packagename}.

A specification file, or specfile, is a data file containing a number of
"specification commands". Each command occurs on a separate line,
and comprises a "keyword",
which is a string (in a close-to-natural language) used to identify a
control parameter, and
an (optional) "value", which defines the value to be assigned to the given
control parameter. All keywords and values are case insensitive,
keywords may be preceded by one or more blanks but
values must not contain blanks, and
each value must be separated from its keyword by at least one blank.
Values must not contain more than 30 characters, and
each line of the specfile is limited to 80 characters,
including the blanks separating keyword and value.

The portion of the specification file used by
{\tt \packagename\_read\_specfile}
must start
with a "{\tt BEGIN \packagename}" command and end with an
"{\tt END}" command.  The syntax of the specfile is thus defined as follows:
\begin{verbatim}
  ( .. lines ignored by FDC_read_specfile .. )
    BEGIN FDC
       keyword    value
       .......    .....
       keyword    value
    END
  ( .. lines ignored by FDC_read_specfile .. )
\end{verbatim}
where keyword and value are two strings separated by (at least) one blank.
The ``{\tt BEGIN \packagename}'' and ``{\tt END}'' delimiter command lines
may contain additional (trailing) strings so long as such strings are
separated by one or more blanks, so that lines such as
\begin{verbatim}
    BEGIN FDC SPECIFICATION
\end{verbatim}
and
\begin{verbatim}
    END FDC SPECIFICATION
\end{verbatim}
are acceptable. Furthermore,
between the
``{\tt BEGIN \packagename}'' and ``{\tt END}'' delimiters,
specification commands may occur in any order.  Blank lines and
lines whose first non-blank character is {\tt !} or {\tt *} are ignored.
The content
of a line after a {\tt !} or {\tt *} character is also
ignored (as is the {\tt !} or {\tt *}
character itself). This provides an easy manner to "comment out" some
specification commands, or to comment specific values
of certain control parameters.

The value of a control parameters may be of three different types, namely
integer, logical or real.
Integer and real values may be expressed in any relevant Fortran integer and
floating-point formats (respectively). Permitted values for logical
parameters are "{\tt ON}", "{\tt TRUE}", "{\tt .TRUE.}", "{\tt T}",
"{\tt YES}", "{\tt Y}", or "{\tt OFF}", "{\tt NO}",
"{\tt N}", "{\tt FALSE}", "{\tt .FALSE.}" and "{\tt F}".
Empty values are also allowed for
logical control parameters, and are interpreted as "{\tt TRUE}".

The specification file must be open for
input when {\tt \packagename\_read\_specfile}
is called, and the associated device number
passed to the routine in device (see below).
Note that the corresponding
file is {\tt REWIND}ed, which makes it possible to combine the specifications
for more than one program/routine.  For the same reason, the file is not
closed by {\tt \packagename\_read\_specfile}.

\subsubsection{To read control parameters from a specification file}
\label{readspec}

Control parameters may be read from a file as follows:
\hskip0.5in
\def\baselinestretch{0.8} {\tt \begin{verbatim}
     CALL FDC_read_specfile( control, device )
\end{verbatim}}
\def\baselinestretch{1.0}

\begin{description}
\itt{control} is a scalar \intentinout argument of type
{\tt \packagename\_control\_type}
(see Section~\ref{typecontrol}).
Default values should have already been set, perhaps by calling
{\tt \packagename\_initialize}.
On exit, individual components of {\tt control} may have been changed
according to the commands found in the specfile. Specfile commands and
the component (see Section~\ref{typecontrol}) of {\tt control}
that each affects are given in Table~\ref{specfile}.

\bctable{|l|l|l|}
\hline
  command & component of {\tt control} & value type \\
\hline
  {\tt error-printout-device} & {\tt \%error} & integer \\
  {\tt printout-device} & {\tt \%out} & integer \\
  {\tt print-level} & {\tt \%print\_level} & integer \\
%  {\tt initial-integer-workspace} & {\tt \%indmin} & integer \\
%  {\tt initial-real-workspace} & {\tt \%valmin} & integer \\
  {\tt maximum-permitted-infeasibility} & {\tt \%max\_infeas} & real \\
  {\tt pivot-tolerance-used-for-dependencies} & {\tt \%pivot\_tol} & real \\
%  {\tt zero-pivot-tolerance} & {\tt \%zero\_pivot} & real \\
  {\tt use-sls}   & {\tt \%use\_sls} & logical \\
  {\tt scale-A}   & {\tt \%scale} & logical \\
  {\tt space-critical}   & {\tt \%space\_critical} & logical \\
  {\tt deallocate-error-fatal}   & {\tt \%deallocate\_error\_fatal} & logical \\
  {\tt symmetric-linear-equation-solver} & {\tt \%symmetric\_linear\_solver} & character \\
 {\tt output-line-prefix} & {\tt \%prefix} & character \\
\hline
\ectable{\label{specfile}Specfile commands and associated
components of {\tt control}.}

\itt{device} is a scalar \intentin argument of type default \integer,
that must be set to the unit number on which the specfile
has been opened. If {\tt device} is not open, {\tt control} will
not be altered and execution will continue, but an error message
will be printed on unit {\tt control\%error}.

\end{description}

%%%%%%%%%%%%%%%%%%%%%% Information printed %%%%%%%%%%%%%%%%%%%%%%%%

\galinfo
If {\tt control\%print\_level} is positive, information about the progress
of the algorithm will be printed on unit {\tt control\-\%out}.
If {\tt control\%print\_level} $= 1$, basic statistics of the performance of the
package will be produced.
If {\tt control\-\%print\_level} $\geq 2$ this
output will be increased to provide details such as the size of each
neglected pivot.

%%%%%%%%%%%%%%%%%%%%%% GENERAL INFORMATION %%%%%%%%%%%%%%%%%%%%%%%%

\galgeneral

\galcommon None.
\galworkspace Provided automatically by the module.
\galroutines None.
\galmodules {\tt \packagename\_find\_dependent} calls the \galahad\ packages
{\tt GALAHAD\_CLOCK},
{\tt GALAHAD\_SY\-M\-BOLS},
{\tt GALAHAD\_STRING},
{\tt GALAHAD\_SMT}, {\tt GALAHAD\_\-ROOTS}, {\tt GALAHAD\_\-SLS},
{\tt GALAHAD\_\-ULS}, {\tt GALAHAD\_SPECFILE} and {\tt GALAHAD\_SPACE}.
\galio Output is under control of the arguments
 {\tt control\%error}, {\tt control\%out} and {\tt control\%print\_level}.
\galrestrictions {\tt n} $\geq 0$, {\tt m} $\geq  0$.
\galportability ISO Fortran~95 + TR 15581 or Fortran~2003.
The package is thread-safe.

%%%%%%%%%%%%%%%%%%%%%% METHOD %%%%%%%%%%%%%%%%%%%%%%%%

\galmethod

A choice of two methods is available. In the first, the matrix
\disp{ \bmK = \mat{cc}{ \alpha \bmI & \bmA^T \\ \bmA & \bmzero } }
is formed and factorized for some small $\alpha > 0$ using the
\libraryname\ package {\tt SLS}---the
factors $\bmK = \bmP \bmL \bmD \bmL^T \bmP^T$ are used to determine
whether $\bmA$ has dependent rows. In particular, in exact arithmetic
dependencies in $\bmA$ will correspond to zero pivots in the block
diagonal matrix $\bmD$.

The second choice of method finds factors
$\bmA = \bmP \bmL \bmU \bmQ$ of the rectangular matrix $\bmA$
using the \libraryname\ package {\tt ULS}.
In this case, dependencies in $\bmA$ will be reflected in zero diagonal
entries in $\bmU$ in exact arithmetic.

The factorization in either case may also be used to
determine whether the system is consistent.

%%%%%%%%%%%%%%%%%%%%%% EXAMPLE %%%%%%%%%%%%%%%%%%%%%%%%

\galexample
Suppose we wish to find whether the linear constraints
$x_1 + 2 x_2 + 3 x_3 + 4 x_4 = 5, 2 x_1 - 4 x_2 + 6 x_3 - 8 x_4 = 10
\;\; \mbox{and} \;\; 5 x_2 + 10 x_4 = 0$
are consistent but redundant.
Then we may use the following code.

{\tt \small
\VerbatimInput{\packageexample}
}
\noindent
This produces the following output:
{\tt \small
\VerbatimInput{\packageresults}
}

\end{document}







