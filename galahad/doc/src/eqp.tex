\documentclass{galahad}

% set the package name

\newcommand{\package}{eqp}
\newcommand{\packagename}{EQP}
\newcommand{\fullpackagename}{\libraryname\_\packagename}

\begin{document}

\galheader

%%%%%%%%%%%%%%%%%%%%%% SUMMARY %%%%%%%%%%%%%%%%%%%%%%%%

\galsummary
This package uses an iterative method
to solve the {\bf equality-constrained quadratic programming problem}
\eqn{qp}{\mbox{minimize}\;\; \half \bmx^T \bmH \bmx + \bmg^T \bmx + f}
subject to the linear constraints
\eqn{lc}{\bmA \bmx  + \bmc = \bmzero,}
where the $n$ by $n$ symmetric matrix $\bmH$, the $m$ by $n$ matrix $\bmA$, 
the  vectors $\bmg$ and $\bmc$, and the scalar $f$ are given.
Full advantage is taken of any zero coefficients in the matrices $\bmH$ 
and $\bmA$.

The package may alternatively be used to minimize the (shifted) 
squared-least-distance objective
\eqn{lsqp}{\half \sum_{j=1}^n w_j^2 ( x_j^{ } - x_j^0 )^2 + \bmg^T \bmx + f,}
subject to the linear constraints \req{lc}, 
for given vectors $\bmw$ and $\bmx^{0}$.

%%%%%%%%%%%%%%%%%%%%%% attributes %%%%%%%%%%%%%%%%%%%%%%%%

\galattributes
\galversions{\tt  \fullpackagename\_single, \fullpackagename\_double}.
\galuses 
{\tt GALAHAD\_\-CLOCK},
{\tt GALAHAD\_SY\-M\-BOLS}, 
{\tt GALAHAD\-\_\-SPACE}, 
{\tt GALAHAD\_QPD},
{\tt GALAHAD\_QPT},
{\tt GALAHAD\_FDC},
{\tt GALAHAD\_SBLS}, 
{\tt GALAHAD\_GLTR}, 
{\tt GALAHAD\_SPECFILE}.
\galdate March 2006.
\galorigin N. I. M. Gould, Rutherford Appleton Laboratory.
\gallanguage Fortran~95 + TR 15581 or Fortran~2003. 
\galparallelism Some options may use OpenMP and its runtime library.

%%%%%%%%%%%%%%%%%%%%%% HOW TO USE %%%%%%%%%%%%%%%%%%%%%%%%

\galhowto

%\subsection{Calling sequences}

Access to the package requires a {\tt USE} statement such as

\medskip\noindent{\em Single precision version}

\hspace{8mm} {\tt USE \fullpackagename\_single}

\medskip\noindent{\em Double precision version}

\hspace{8mm} {\tt USE  \fullpackagename\_double}

\medskip

\noindent
If it is required to use both modules at the same time, the derived types 
{\tt QPT\_problem\_type}, 
{\tt \packagename\_time\_type}, 
{\tt \packagename\_\-control\-\_type}, 
{\tt \packagename\_inform\_type} 
and
{\tt \packagename\_data\_type}
(Section~\ref{galtypes})
and the subroutines
{\tt \packagename\_initialize}, 
{\tt \packagename\_\-solve},
{\tt \packagename\_\-resolve},
{\tt \packagename\_\-terminate},
(Section~\ref{galarguments})
and 
{\tt \packagename\_read\_specfile}
(Section~\ref{galfeatures})
must be renamed on one of the {\tt USE} statements.

%%%%%%%%%%%%%%%%%%%%%% matrix formats %%%%%%%%%%%%%%%%%%%%%%%%

\galmatrix
Both the Hessian matrix $\bmH$ and the constraint Jacobian $\bmA$
may be stored in a variety of input formats.

\subsubsection{Dense storage format}\label{dense}
The matrix $\bmA$ is stored as a compact 
dense matrix by rows, that is, the values of the entries of each row in turn are
stored in order within an appropriate real one-dimensional array.
Component $n \ast (i-1) + j$ of the storage array {\tt A\%val} will hold the 
value $a_{ij}$ for $i = 1, \ldots , m$, $j = 1, \ldots , n$.
Since $\bmH$ is symmetric, only the lower triangular part (that is the part 
$h_{ij}$ for $1 \leq j \leq i \leq n$) need be held. In this case
the lower triangle will be stored by rows, that is 
component $i \ast (i-1)/2 + j$ of the storage array {\tt H\%val}  
will hold the value $h_{ij}$ (and, by symmetry, $h_{ji}$)
for $1 \leq j \leq i \leq n$.

\subsubsection{Sparse co-ordinate storage format}\label{coordinate}
Only the nonzero entries of the matrices are stored. For the 
$l$-th entry of $\bmA$, its row index $i$, column index $j$ 
and value $a_{ij}$
are stored in the $l$-th components of the integer arrays {\tt A\%row}, 
{\tt A\%col} and real array {\tt A\%val}, respectively.
The order is unimportant, but the total
number of entries {\tt A\%ne} is also required. 
The same scheme is applicable to
$\bmH$ (thus requiring integer arrays {\tt H\%row}, {\tt H\%col}, a real array 
{\tt H\%val} and an integer value {\tt H\%ne}),
except that only the entries in the lower triangle need be stored.

\subsubsection{Sparse row-wise storage format}\label{rowwise}
Again only the nonzero entries are stored, but this time
they are ordered so that those in row $i$ appear directly before those
in row $i+1$. For the $i$-th row of $\bmA$, the $i$-th component of a 
integer array {\tt A\%ptr} holds the position of the first entry in this row,
while {\tt A\%ptr} $(m+1)$ holds the total number of entries plus one.
The column indices $j$ and values $a_{ij}$ of the entries in the $i$-th row 
are stored in components 
$l =$ {\tt A\%ptr}$(i)$, \ldots ,{\tt A\%ptr} $(i+1)-1$ of the 
integer array {\tt A\%col}, and real array {\tt A\%val}, respectively. 
The same scheme is applicable to
$\bmH$ (thus requiring integer arrays {\tt H\%ptr}, {\tt H\%col}, and 
a real array {\tt H\%val}),
except that only the entries in the lower triangle need be stored.

For sparse matrices, this scheme almost always requires less storage than 
its predecessor.

\subsubsection{Diagonal storage format}\label{diagonal}
If $\bmH$ is diagonal (i.e., $h_{ij} = 0$ for all $1 \leq i \neq j \leq n$)
only the diagonals entries $h_{ii}$, $1 \leq i \leq n$,  need be stored,
and the first $n$ components of the array {\tt H\%val} may be used for 
the purpose. There is no sensible equivalent for the non-square $\bmA$.

%%%%%%%%%%%%%%%%%%%%%% long integers %%%%%%%%%%%%%%%%%%%%%%

\subsection{Integer kinds}\label{Integer kinds}
We use the term
long \integer\ to denote {\tt INTEGER\-(kind=long)}, where 
{\tt long = selected\_int\_kind(18))}.

%%%%%%%%%%%%%%%%%%%%%% OpenMP usage %%%%%%%%%%%%%%%%%%%%%%%%

\subsection{OpenMP}
OpenMP may be used by the {\tt \fullpackagename} package to provide 
parallelism for some solver options in shared memory environments.  
See the documentation for the \galahad\ package {\tt SLS} for more details.
To run in parallel, OpenMP 
must be enabled at compilation time by using the correct compiler flag 
(usually some variant of {\tt -openmp}). 
The number of threads may be controlled at runtime
by setting the environment variable {\tt OMP\_NUM\_THREADS}.

\noindent
The code may be compiled and run in serial mode.

%%%%%%%%%%%%%%%%%%%%%% derived types %%%%%%%%%%%%%%%%%%%%%%%%

\galtypes
Six derived data types are accessible from the package.

%%%%%%%%%%% matrix data type %%%%%%%%%%%

\subsubsection{The derived data type for holding matrices}\label{typesmt}
The derived data type {\tt SMT\_TYPE} is used to hold the matrices $\bmA$
and $\bmH$. The components of {\tt SMT\_TYPE} used here are:

\begin{description}

\ittf{m} is a scalar component of type default \integer, 
that holds the number of rows in the matrix. 
 
\ittf{n} is a scalar component of type default \integer, 
that holds the number of columns in the matrix. 
 
\ittf{ne} is a scalar variable of type default \integer, that
holds the number of matrix entries.

\ittf{type} is a rank-one allocatable array of type default \character, that
is used to indicate the matrix storage scheme used. Its precise length and
content depends on the type of matrix to be stored (see \S\ref{typeprob}).

\ittf{val} is a rank-one allocatable array of type default \realdp\, 
and dimension at least {\tt ne}, that holds the values of the entries. 
Each pair of off-diagonal entries $h_{ij} = h_{ji}$ of a {\em symmetric}
matrix $\bmH$ is represented as a single entry 
(see \S\ref{dense}--\ref{rowwise}).
Any duplicated entries that appear in the sparse 
co-ordinate or row-wise schemes will be summed. 

\ittf{row} is a rank-one allocatable array of type default \integer, 
and dimension at least {\tt ne}, that may hold the row indices of the entries. 
(see \S\ref{coordinate}).

\ittf{col} is a rank-one allocatable array of type default \integer, 
and dimension at least {\tt ne}, that may hold the column indices of the entries
(see \S\ref{coordinate}--\ref{rowwise}).

\ittf{ptr} is a rank-one allocatable array of type default \integer, 
and dimension at least {\tt m + 1}, that may hold the pointers to
the first entry in each row (see \S\ref{rowwise}).

\end{description}

%%%%%%%%%%% problem type %%%%%%%%%%%

\subsubsection{The derived data type for holding the problem}\label{typeprob}
The derived data type {\tt QPT\_problem\_type} is used to hold 
the problem. The components of 
{\tt QPT\_problem\_type} 
are:

\begin{description}

\ittf{n} is a scalar variable of type default \integer, 
 that holds the number of optimization variables, $n$.  
              
\ittf{m} is a scalar variable of type default \integer, 
 that holds the number of linear constraints, $m$.
              
\itt{Hessian\_kind} is a scalar variable of type default \integer, 
that is used to indicate what type of Hessian the problem involves.
Possible values for {\tt Hessian\_kind} are:

\begin{description}
\itt{<0}  In this case, a general quadratic program of the form
\req{qp} is given. The Hessian matrix $\bmH$ will be provided in the 
component {\tt H} (see below).

\itt{0}  In this case, a linear program, that is a problem of the form 
\req{lsqp} with weights $\bmw = 0$, is given.

\itt{1} In this case, a least-distance problem of the form \req{lsqp}
with weights $w_{j} = 1$ for $j = 1, \ldots , n$ is given.

\itt{>1} In this case, a weighted least-distance problem of the form \req{lsqp}
with general weights $\bmw$ is given. The weights will be
provided in the component {\tt WEIGHT} (see below).
\end{description}
By default {\tt Hessian\_kind = - 1}.

\ittf{H} is scalar variable of type {\tt SMT\_TYPE} 
that holds the Hessian matrix $\bmH$. The following components
are used:

\begin{description}

\itt{H\%type} is an allocatable array of rank one and type default \character, that
is used to indicate the storage scheme used. If the dense storage scheme 
(see Section~\ref{dense}) is used, 
the first five components of {\tt H\%type} must contain the
string {\tt DENSE}.
For the sparse co-ordinate scheme (see Section~\ref{coordinate}), 
the first ten components of {\tt H\%type} must contain the
string {\tt COORDINATE},  
for the sparse row-wise storage scheme (see Section~\ref{rowwise}),
the first fourteen components of {\tt H\%type} must contain the
string {\tt SPARSE\_BY\_ROWS},
and for the diagonal storage scheme (see Section~\ref{diagonal}),
the first eight components of {\tt H\%type} must contain the
string {\tt DIAGONAL}.

For convenience, the procedure {\tt SMT\_put} 
may be used to allocate sufficient space and insert the required keyword
into {\tt H\%type}.
For example, if {\tt prob} is of derived type {\tt \packagename\_problem\_type}
and involves a Hessian we wish to store using the co-ordinate scheme,
we may simply
%\vspace*{-2mm}
{\tt 
\begin{verbatim}
        CALL SMT_put( prob%H%type, 'COORDINATE', istat )
\end{verbatim}
}
%\vspace*{-4mm}
\noindent
See the documentation for the \galahad\ package {\tt SMT} 
for further details on the use of {\tt SMT\_put}.

\itt{H\%ne} is a scalar variable of type default \integer, that 
holds the number of entries in the {\bf lower triangular} part of $\bmH$
in the sparse co-ordinate storage scheme (see Section~\ref{coordinate}). 
It need not be set for any of the other three schemes.

\itt{H\%val} is a rank-one allocatable array of type default \realdp, that holds
the values of the entries of the {\bf lower triangular} part
of the Hessian matrix $\bmH$ in any of the 
storage schemes discussed in Section~\ref{galmatrix}.

\itt{H\%row} is a rank-one allocatable array of type default \integer,
that holds the row indices of the {\bf lower triangular} part of $\bmH$ 
in the sparse co-ordinate storage
scheme (see Section~\ref{coordinate}). 
It need not be allocated for any of the other three schemes.

\itt{H\%col} is a rank-one allocatable array variable of type default \integer,
that holds the column indices of the {\bf lower triangular} part of 
$\bmH$ in either the sparse co-ordinate 
(see Section~\ref{coordinate}), or the sparse row-wise 
(see Section~\ref{rowwise}) storage scheme.
It need not be allocated when the dense or diagonal storage schemes are used.

\itt{H\%ptr} is a rank-one allocatable array of dimension {\tt n+1} and type 
default \integer, that holds the starting position of 
each row of the {\bf lower triangular} part of $\bmH$, as well
as the total number of entries plus one, in the sparse row-wise storage
scheme (see Section~\ref{rowwise}). It need not be allocated when the
other schemes are used.

\end{description}
If {\tt Hessian\_kind} $\geq 0$, the components of {\tt H} need not be set.

\itt{WEIGHT} is a rank-one allocatable array type default \realdp, that 
should be allocated to have length {\tt n}, and its $j$-th component 
filled with the value $w_{j}$ for $j = 1, \ldots , n$, 
whenever {\tt Hessian\_kind} $>1$.
If {\tt Hessian\_kind} $\leq 1$, {\tt WEIGHT} need not be allocated.

\itt{target\_kind} is a scalar variable of type default \integer, 
that is used to indicate whether the components of the targets $\bmx^0$ 
(if they are used) have special or general values. Possible values for 
{\tt target\_kind} are:
\begin{description}
\itt{0}  In this case, $\bmx^0 = 0$.

\itt{1} In this case, $x^0_{j} = 1$ for $j = 1, \ldots , n$.

\itt{$\neq$ 0,1} In this case, general values of $\bmx^0$ will be used,
     and will be provided in the component {\tt X0} (see below).
\end{description}
By default {\tt target\_kind = - 1}.

\ittf{X0} is a rank-one allocatable array type default \realdp, that 
should be allocated to have length {\tt n}, and its $j$-th component 
filled with the value $x_{j}^0$ for $j = 1, \ldots , n$, 
whenever {\tt Hessian\_kind} $>0$ and {\tt target\_kind} $\neq 0,1$.
If {\tt Hessian\_kind} $\leq 0$ or {\tt target\_kind} $= 0,1$,
{\tt X0} need not be allocated.

\itt{gradient\_kind} is a scalar variable of type default \integer, 
that is used to indicate whether the components of the gradient $\bmg$ 
have special or general values. Possible values for {\tt gradient\_kind} are:
\begin{description}
\itt{0}  In this case, $\bmg = 0$.

\itt{1} In this case, $g_{j} = 1$ for $j = 1, \ldots , n$.

\itt{$\neq$ 0,1} In this case, general values of $\bmg$ will be used,
     and will be provided in the component {\tt G} (see below).
\end{description}
By default {\tt gradient\_kind = - 1}.

\ittf{G} is a rank-one allocatable array of dimension {\tt n} and type 
default \realdp, that holds the gradient $\bmg$ 
of the linear term of the quadratic objective function.
The $j$-th component of 
{\tt G}, $j = 1,  \ldots ,  n$, contains $\bmg_{j}$.
If {\tt gradient\_kind} {= 0, 1}, {\tt G} need not be allocated.

\ittf{f} is a scalar variable of type default \realdp, that holds 
the constant term, $f$, in the objective function.

\ittf{A} is scalar variable of type {\tt SMT\_TYPE} 
that holds the Jacobian matrix $\bmA$. The following components are used:

\begin{description}

\itt{A\%type} is an allocatable array of rank one and type default \character, that
is used to indicate the storage scheme used. If the dense storage scheme 
(see Section~\ref{dense}) is used, 
the first five components of {\tt A\%type} must contain the
string {\tt DENSE}.
For the sparse co-ordinate scheme (see Section~\ref{coordinate}), 
the first ten components of {\tt A\%type} must contain the
string {\tt COORDINATE}, while 
for the sparse row-wise storage scheme (see Section~\ref{rowwise}),
the first fourteen components of {\tt A\%type} must contain the
string {\tt SPARSE\_BY\_ROWS}.

Just as for {\tt H\%type} above, the procedure {\tt SMT\_put} 
may be used to allocate sufficient space and insert the required keyword
into {\tt A\%type}.
Once again, if {\tt prob} is of derived type {\tt \packagename\_problem\_type}
and involves a Jacobian we wish to store using the sparse row-wise 
storage scheme, we may simply
%\vspace*{-2mm}
{\tt 
\begin{verbatim}
        CALL SMT_put( prob%A%type, 'SPARSE_BY_ROWS', istat )
\end{verbatim}
}
%\vspace*{-4mm}
\noindent

\itt{A\%ne} is a scalar variable of type default \integer, that 
holds the number of entries in $\bmA$
in the sparse co-ordinate storage scheme (see Section~\ref{coordinate}). 
It need not be set for either of the other two schemes.

\itt{A\%val} is a rank-one allocatable array of type default \realdp, that holds
the values of the entries of the Jacobian matrix $\bmA$ in any of the 
storage schemes discussed in Section~\ref{galmatrix}.

\itt{A\%row} is a rank-one allocatable array of type default \integer,
that holds the row indices of $\bmA$ in the sparse co-ordinate storage
scheme (see Section~\ref{coordinate}). 
It need not be allocated for either of the other two schemes.

\itt{A\%col} is a rank-one allocatable array variable of type default \integer,
that holds the column indices of $\bmA$ in either the sparse co-ordinate 
(see Section~\ref{coordinate}), or the sparse row-wise 
(see Section~\ref{rowwise}) storage scheme.
It need not be allocated when the dense storage scheme is used.

\itt{A\%ptr} is a rank-one allocatable array of dimension {\tt m+1} and type 
default \integer, that holds the 
starting position of each row of $\bmA$, as well
as the total number of entries plus one, in the sparse row-wise storage
scheme (see Section~\ref{rowwise}). It need not be allocated when the
other schemes are used.

\end{description}

\ittf{C} is a rank-one allocatable array of dimension {\tt m} and type default 
\realdp, that holds the values of the vector$\bmc$ of constant terms
for the constraints.
The $i$-th component of {\tt C}, $i = 1,  \ldots ,  m$, contains 
$c_{i}$.  

\ittf{X} is a rank-one allocatable array of dimension {\tt n} and type 
default \realdp, 
that holds the values $\bmx$ of the optimization variables.
The $j$-th component of {\tt X}, $j = 1,  \ldots , n$, contains $x_{j}$.  

\ittf{Y} is a rank-one allocatable array of dimension {\tt m} and type 
default \realdp, that holds
the values $\bmy$ of estimates  of the Lagrange multipliers
corresponding to the linear constraints (see Section~\ref{galmethod}).
The $i$-th component of {\tt Y}, $i = 1,  \ldots ,  m$, contains $y_{i}$.  

\end{description}

%%%%%%%%%%% control type %%%%%%%%%%%

\subsubsection{The derived data type for holding control 
 parameters}\label{typecontrol}
The derived data type 
{\tt \packagename\_control\_type} 
is used to hold controlling data. Default values may be obtained by calling 
{\tt \packagename\_initialize}
(see Section~\ref{subinit}),
while components may also be changed by calling 
{\tt \fullpackagename\_read\-\_spec}
(see Section~\ref{readspec}). 
The components of 
{\tt \packagename\_control\_type} 
are:

\begin{description}

\itt{error} is a scalar variable of type default \integer, that holds the
stream number for error messages. Printing of error messages in 
{\tt \packagename\_solve} and {\tt \packagename\_terminate} 
is suppressed if {\tt error} $\leq 0$.
The default is {\tt error = 6}.

\ittf{out} is a scalar variable of type default \integer, that holds the
stream number for informational messages. Printing of informational messages in 
{\tt \packagename\_solve} is suppressed if {\tt out} $< 0$.
The default is {\tt out = 6}.

\itt{print\_level} is a scalar variable of type default \integer, that is used
to control the amount of informational output which is required. No 
informational output will occur if {\tt print\_level} $\leq 0$. If 
{\tt print\_level} $= 1$, a single line of output will be produced for each
iteration of the process. If {\tt print\_level} $\geq 2$, this output will be
increased to provide significant detail of each iteration.
The default is {\tt print\_level = 0}.

%\itt{preconditioner} is a scalar variable of type default \integer, 
%that specifies which preconditioner to be used. When finding
%the required minimizer, the system matrix
%\disp{ \bmK_{H} = \mat{cc}{ \bmH & \bmA^T \\ \bmA  & \bmzero }}
%will be replaced by a so-called preconditioner of the form
%\eqn{prec1}{ \bmK_{G} = \mat{cc}{ \bmG & \bmA^T \\ \bmA  & \bmzero}
%\equiv \bmP \mat{ccc}{ \bmG_{11}^{} & \bmG_{21}^T & \bmA_1^T \\
%\bmG_{21}^{} & \bmG_{22}^{} & \bmA_2^T \\
%\bmA_{1}^{} & \bmA_{2}^{} & \bmzero} \bmP^T.}
%The leading-block matrix $\bmG$ will be a suitably-chosen 
%approximation to $\bmH$ for which $\bmK_{G}$ is easier to invert/factor\-ize
%than $\bmK_{H}$, while $\bmP$ is an appropriately-chosen permutation matrix.
%Possible values of {\tt preconditioner} are:

%\begin{description}
%\itt{0} the preconditioner is chosen automatically on the basis of which option 
%        looks likely to be the most efficient.
%\itt{1} $\bmG$ is chosen to be the identity matrix.
%\itt{2} $\bmG$ is chosen to be $\bmH$
%\itt{3} $\bmG$ is chosen to be the diagonal matrix whose diagonals
%        are the larger of those of $\bmH$ and a positive constant
%        (see {\tt min\_diagonal} below).
%\itt{4} $\bmG$ is chosen to be the band matrix  of given semi-bandwidth
%        whose entries coincide with those of $\bmH$ within the band.
%        (see {\tt semi\_bandwidth} below).
%\itt{11} $\bmG$ is chosen so that $\bmG_{11} = 0$, $\bmG_{21} = 0$
%        and $\bmG_{22} = \bmH_{22}$.
%\itt{12} $\bmG$ is chosen so that $\bmG_{11} = 0$, $\bmG_{21} = \bmH_{21}$
%        and $\bmG_{22} = \bmH_{22}$.
%\itt{-1} $\bmG$ is chosen so that $\bmG_{11} = 0$, $\bmG_{21} = 0$,
%        $\bmG_{22}$ is the identity matrix, and the preconditioner is computed
%        implicitly.
%\itt{-2} $\bmG$ is chosen so that $\bmG_{11} = 0$, $\bmG_{21} = 0$,
%        $\bmG_{22} = \bmH_{22}$ and the preconditioner is computed
%        implicitly.
%\end{description}
%Other values may be added in future.
%The default is {\tt preconditioner = 0}.

%\itt{min\_diagonal} is a scalar variable of type default \realdp, that 
%specifies the smallest value of the diagonal when diagonal preconditioning,
%{\tt precon = 3}, is used. The default is {\tt min\_diagonal = 0.00001}.

%\itt{semi\_bandwidth} is a scalar variable of type default \integer, that 
%specifies the semi-bandwidth of the band preconditioner when {\tt precon = 4}, 
%if appropriate.
%The default is {\tt semi\_bandwidth = 5}.

%\itt{factorization} is a scalar variable of type default \integer, that 
%specifies which factorization of the preconditioner should be used.  
%Possible values are:

%\begin{description}
%\itt{0} the factorization is chosen automatically on the basis of which option 
%        looks likely to be the most efficient.
%\itt{1} a Schur-complement factorization, involving factors of 
%       $\bmG$ and $\bmA \bmG^{-1} \bmA^T$, will be used.
%\itt{2} an augmented-system factorization, involving factors of $\bmK_G$,
%        will be used.
%\end{description}
%The default is {\tt factorization = 0}.

%\itt{max\_col} is a scalar variable of type default \integer, that specifies
%the maximum number of nonzeros in a column of $\bmA$ which is permitted
%by the Schur-complement factorization.
%The default is {\tt max\_col = 35}.

%\itt{indmin} is a scalar variable of type default \integer, that specifies
%an initial estimate as to the amount of integer workspace required by 
%the factorization package {\tt SILS}.
%The default is {\tt indmin = 1000}.

%\itt{valmin} is a scalar variable of type default \integer, that specifies
%an initial estimate as to the amount of real workspace required by 
%the factorization package {\tt SILS}.
%The default is {\tt valmin = 1000}.

%\itt{len\_ulsmin} is a scalar variable of type default \integer, that specifies
%an initial estimate as to the amount of workspace required by 
%the factorization package {\tt ULS}.
%The default is {\tt len\_ulsmin = 1000}.

%\itt{itref\_max} is a scalar variable of type default \integer, that specifies 
%the maximum number of iterative refinements allowed with each application 
%of the preconditioner.
%The default is {\tt itref\_max = 1}.

\itt{new\_h} is a scalar variable of type default \integer, that is used
to indicate how $\bmH$ has changed (if at all) since the previous call
to {\tt \packagename\_solve}. Possible values are:
\begin{description}
\itt{0} $\bmH$ is unchanged
\itt{1} the values in $\bmH$ have changed, but its nonzero structure 
is as before.
\itt{2} both the values and structure of $\bmH$ have changed.
\end{description}
The default is {\tt new\_h = 2}.

\itt{new\_a} is a scalar variable of type default \integer, that is used
to indicate how $\bmA$ has changed (if at all) since the previous call
to {\tt \packagename\_solve}. Possible values are:
\begin{description}
\itt{0} $\bmA$ is unchanged
\itt{1} the values in $\bmA$ have changed, but its nonzero structure 
is as before.
\itt{2} both the values and structure of $\bmA$ have changed.
\end{description}
The default is {\tt new\_a = 2}.

\itt{cg\_maxit} is a scalar variable of type default \integer, that is used
to limit the number of conjugate-gradient iterations performed in 
the optimality phase. If 
{\tt cg\_maxit} is negative, no limit will be impossed. The default
is {\tt cg\_maxit = 200}.

%\itt{pivot\_tol}  is a scalar variable of type default 
%\realdp, that holds the
%threshold pivot tolerance used by the matrix factorization.  See 
%the documentation for the packages {\tt SILS} and {\tt ULS} for details.
%The default is {\tt pivot\_tol = 0.01}.

%\itt{pivot\_tol\_for\_basis} is a scalar variable of type default 
%\realdp, that holds the
%threshold pivot  tolerance used by the package {\tt ULS} 
%when computing the non-singular basis matrix $\bmA_1$ for
%an implicit preconditioner. Since the calculation of a
%suitable basis is crucial, it is sensible to pick a larger
%value of {\tt pivot\_tol\_for\_basis} than of {\tt pivot\_tol}.
%The default is {\tt pivot\_tol\_for\_basis = 0.5}.

%\itt{zero\_pivot} is a scalar variable of type default \realdp, 
%that is used to detect singularity. Any pivot encountered during the
%factorization whose absolute value is less than or equal to  {\tt zero\_pivot}
%will be regarded as zero, and the matrix as singular.
%The default is {\tt zero\_pivot =} $u^{0.75}$
%where $u$ is {\tt EPSILON(1.0)} ({\tt EPSILON(1.0D0)} in 
%{\tt \fullpackagename\_double}).

\itt{radius} is a scalar variable of type default 
\realdp, that may be used to specify an upper bound on the norm
of the allowed solution (a ``trust-region'' constraint) during the
iterative solution of the optimality phase of the problem.
This is particularly useful if the
problem is unbounded from below. If {\tt radius} is set too small, there
is a possibility that this will preclude the package from 
finding the actual solution.
If {\tt initial\_radius} is not positive, it will be reset to the 
default value, {\tt initial\_radius = SQRT(0.1*HUGE(1.0))}
({\tt SQRT(0.1*HUGE(1.0D0))} in {\tt \fullpackagename\_double}).

%\itt{inner\_fraction\_opt} is a scalar variable of type default 
%\realdp, that specifies the
%fraction of the optimal value which is acceptable for the 
%iterative solution of the optimality phase of the problem using the package 
%{\tt \libraryname\_GLTR},
%and correspond to the value {\tt control\%inner\_fraction} in that package.
%A negative value is considered to be zero, and a value of larger than one
%is considered to be one. Reducing {\tt fraction\_opt} below one will result
%in a reduction of the computation performed at the expense of an inferior
%approximation to the optimal value.
%The default is {\tt inner\_fraction\_opt = 0.1}.

\itt{inner\_stop\_relative} and {\tt inner\_stop\_absolute} 
are scalar variables of type default \realdp, 
that hold the relative and absolute convergence tolerances for the
iterative solution of the optimality phase of the problem using the package 
{\tt \libraryname\_GLTR},
and correspond to the values {\tt control\%stop\_relative} and
{\tt control\%stop\_absolute} in that package.
The defaults are 
%{\tt inner\_stop\_relative = 0.0}
{\tt inner\_stop\_relative = 0.01}
and \sloppy {\tt inner\_stop\_absolute =} $\sqrt{u}$,
where $u$ is {\tt EPSILON(1.0)} ({\tt EPSILON(1.0D0)} in 
{\tt \fullpackagename\_double}).

\itt{max\_infeasibility\_relative} and {\tt max\_infeasibility\_absolute} 
are scalar variables of type default \realdp, 
that hold the relative and absolute tolerances for assessing 
infeasibility in the feasibility phase.  If the constraints are believed to be 
rank defficient and the norm of the residual $\bmA \bmx_T + \bmc$
at a "typical" feasiblke point is larger than
  $\max( ${\tt max\_infeasibility\_relative}$ \ast \| A\|,$ 
{\tt max\_infeasibility\_absolute}$)$,
the problem will be marked as infeasible,
The defaults are 
{\tt max\_infeasibility\_relative = 
\tt max\_infeasibility\_absolute =} $u^{0.75}$
where $u$ is {\tt EPSILON(1.0)} ({\tt EPSILON(1.0D0)} in 
{\tt \fullpackagename\_double}).

\itt{remove\_dependencies} is a scalar variable of type default \logical, 
that must be set \true\ if linear dependent constraints
$\bmA \bmx + \bmc = \bmzero$ should be removed and \false\ otherwise.
The default is {\tt remove\_dependencies = .TRUE.}.

%\itt{find\_basis\_by\_transpose} is a scalar variable of type default \logical, 
%that must be set \true\ if the invertible sub-block $\bmA_1$ of the 
%columns of $\bmA$ is computed by analysing the transpose of $\bmA$ 
%and \false\ if the analysis is based on $\bmA$ itself. Generally
%an analysis based on the transpose is faster.
%The default is {\tt find\_basis\_by\_transpose = .TRUE.}.

\itt{space\_critical} is a scalar variable of type default \logical, 
that must be set \true\ if space is critical when allocating arrays
and  \false\ otherwise. The package may run faster if 
{\tt space\_critical} is \false\ but at the possible expense of a larger
storage requirement. The default is {\tt space\_critical = .FALSE.}.

\itt{deallocate\_error\_fatal} is a scalar variable of type default \logical, 
that must be set \true\ if the user wishes to terminate execution if
a deallocation  fails, and \false\ if an attempt to continue
will be made. The default is {\tt deallocate\_error\_fatal = .FALSE.}.

\itt{FDC\_control} is a scalar variable of type 
{\tt FDC\_control\_type}
whose components are used to control any detection of linear dependencies
performed by the package 
{\tt \libraryname\_FDC}. 
See the specification sheet for the package 
{\tt \libraryname\_FDC} 
for details, and appropriate default values.

\itt{GLTR\_control} is a scalar variable argument of type 
{\tt GLTR\_control\_type} that is used to pass control
options to the conjugate-gradient solver used to solve linear systems that arise. 
See the documentation for the \galahad\ package {\tt GLTR} for further details. 
In particular, default values are as for {\tt GLTR}.

\itt{SBLS\_control} is a scalar variable argument of type 
{\tt SBLS\_control\_type} that is used to pass control
options to the symmetric block linear equation preconditioner 
used to help solve linear systems that arise. 
See the documentation for the \galahad\ package {\tt SBLS} for further details. 
In particular, default values are as for {\tt SBLS}.

\itt{prefix} is a scalar variable of type default \character\
and length 30, that may be used to provide a user-selected 
character string to preface every line of printed output. 
Specifically, each line of output will be prefaced by the string 
{\tt prefix(2:LEN(TRIM( prefix ))-1)},
thus ignoreing the first and last non-null components of the
supplied string. If the user does not want to preface lines by such
a string, they may use the default {\tt prefix = ""}.

\end{description}


%%%%%%%%%%% time type %%%%%%%%%%%

\subsubsection{The derived data type for holding timing 
 information}\label{typetime}
The derived data type 
{\tt \packagename\_time\_type} 
is used to hold elapsed CPU and system clock times for the various parts of 
the calculation. The components of 
{\tt \packagename\_time\_type} 
are:
\begin{description}
\itt{total} is a scalar variable of type default \realdp, that gives
 the total CPU time spent in the package.

\itt{find\_dependent} is a scalar variable of type default \realdp, that gives
 the CPU time spent detecting and removing dependent constraints prior 
to solution.

\itt{factorize} is a scalar variable of type default \realdp, that gives
 the CPU time spent factorizing the required matrices.

\itt{solve} is a scalar variable of type default \realdp, that gives
 the CPU time spent computing the solution given the factorization(s).

\itt{clock\_total} is a scalar variable of type default \realdp, that gives
 the total elapsed system clock time spent in the package.

\itt{clock\_find\_dependent} is a scalar variable of type default \realdp, 
that gives  the elapsed system clock time spent detecting and removing 
dependent constraints prior to solution.

\itt{clock\_factorize} is a scalar variable of type default \realdp, that gives
 the elapsed system clock time spent factorizing the required matrices.

\itt{clock\_solve} is a scalar variable of type default \realdp, that gives
 the elapsed system clock time spent computing the search direction.

\end{description}

%%%%%%%%%%% info type %%%%%%%%%%%

\subsubsection{The derived data type for holding informational
 parameters}\label{typeinform}
The derived data type 
{\tt \packagename\_inform\_type} 
is used to hold parameters that give information about the progress and needs 
of the algorithm. The components of 
{\tt \packagename\_inform\_type} 
are:

\begin{description}

\itt{status} is a scalar variable of type default \integer, that gives the
exit status of the algorithm. 
%See Sections~\ref{galerrors} and \ref{galinfo}
See Section~\ref{galerrors} 
for details.

\itt{alloc\_status} is a scalar variable of type default \integer, that gives
the status of the last attempted array allocation or deallocation.
This will be 0 if {\tt status = 0}.

\itt{bad\_alloc} is a scalar variable of type default \character\
and length 80, that  gives the name of the last internal array 
for which there were allocation or deallocation errors.
This will be the null string if {\tt status = 0}. 

\itt{cg\_iter} is a scalar variable of type default \integer, that gives the
total number of conjugate-gradient iterations required.

\itt{factorization\_integer} is a scalar variable of type long
\integer, that gives the amount of integer storage used for the matrix 
factorization.

\itt{factorization\_real} is a scalar variable of type long \integer, 
that gives the amount of real storage used for the matrix factorization.

\ittf{obj} is a scalar variable of type default \realdp, that holds the
value of the objective function at the best estimate of the solution found.

\ittf{time} is a scalar variable of type {\tt \packagename\_time\_type} 
whose components are used to hold elapsed CPU and system clock times for the 
various parts of the calculation (see Section~\ref{typetime}).

\itt{FDC\_inform} is a scalar variable of type {\tt FDC\_inform\_type}
whose components are used to provide information about any detection of linear 
dependencies performed by the package {\tt \libraryname\_FDC}. See the
specification sheet for the package {\tt \libraryname\_FDC} for details.

\itt{SBLS\_inform} is a scalar variable of type {\tt SBLS\_inform\_type} 
whose components are used to hold information relating to the 
formation and factorization of the preconditioner. See the
documentation for the \galahad\ package {\tt SBLS} for further details.

\itt{GLTR\_inform} is a scalar variable of type {\tt GLTR\_inform\_type} 
whose components are used to hold information relating to the 
computation of the solution via the conjugate-gradient method. See the
documentation for the \galahad\ package {\tt GLTR} for further details.

\end{description}


%%%%%%%%%%% data type %%%%%%%%%%%

\subsubsection{The derived data type for holding problem data}\label{typedata}
The derived data type 
{\tt \packagename\_data\_type} 
is used to hold all the data for a particular problem,
or sequences of problems with the same structure, between calls of 
{\tt \packagename} procedures. 
This data should be preserved, untouched, from the initial call to 
{\tt \packagename\_initialize}
to the final call to
{\tt \packagename\_terminate}.

%%%%%%%%%%%%%%%%%%%%%% argument lists %%%%%%%%%%%%%%%%%%%%%%%%

\galarguments
There are three procedures for user calls
(see Section~\ref{galfeatures} for further features): 

\begin{enumerate}
\item The subroutine 
      {\tt \packagename\_initialize} 
      is used to set default values, and initialize private data, 
      before solving one or more problems with the
      same sparsity and bound structure.
\item The subroutine 
      {\tt \packagename\_solve} 
      is called to solve the problem.
\item The subroutine 
      {\tt \packagename\_terminate} 
      is provided to allow the user to automatically deallocate array 
       components of the private data, allocated by 
       {\tt \packagename\_solve}, 
       at the end of the solution process. 
\end{enumerate}
%We use square brackets {\tt [ ]} to indicate \optional arguments.

%%%%%% initialization subroutine %%%%%%

\subsubsection{The initialization subroutine}\label{subinit}
 Default values are provided as follows:
\vspace*{1mm}

\hspace{8mm}
{\tt CALL \packagename\_initialize( data, control, inform )}

%\vspace*{-3mm}
\begin{description}

\itt{data} is a scalar \intentinout\ argument of type 
{\tt \packagename\_data\_type}
(see Section~\ref{typedata}). It is used to hold data about the problem being 
solved. 

\itt{control} is a scalar \intentout\ argument of type 
{\tt \packagename\_control\_type}
(see Section~\ref{typecontrol}). 
On exit, {\tt control} contains default values for the components as
described in Section~\ref{typecontrol}.
These values should only be changed after calling 
{\tt \packagename\_initialize}.

\itt{inform} is a scalar \intentinout\ argument of type 
{\tt \packagename\_inform\_type}
(see Section~\ref{typeinform}). A successful call to
{\tt \packagename\_initialize}
is indicated when the  component {\tt status} has the value 0. 
For other return values of {\tt status}, see Section~\ref{galerrors}.

\end{description}

%%%%%%%%% main solution subroutine %%%%%%

\subsubsection{The equality-constrained-quadratic programming subroutine}
The equality-constrained quadratic programming algorithm is called as follows:
\vspace*{1mm}

\hspace{8mm}
{\tt CALL \packagename\_solve( p, data, control, inform )}

\vspace*{-3mm}
\begin{description}
\ittf{p} is a scalar \intentinout\ argument of type 
{\tt QPT\_problem\_type}
(see Section~\ref{typeprob}). 
It is used to hold data about the problem being solved.
The user must have allocated all array components,
and set appropriate values for all components.
Users are free to choose whichever
of the matrix formats described in Section~\ref{galmatrix} 
is appropriate for $\bmA$ and $\bmH$ for their application---different 
formats may be used for the two matrices.

The components {\tt p\%X} and {\tt p\%Y}
must be set to initial estimates, $\bmx^{0}$, of the solution variables, 
$\bmx$, and Lagrange multipliers for the constraints, $\bmy$.
Inappropriate initial values will be altered, so the user should
not be overly concerned if suitable values are not apparent, and may be
content with merely setting {\tt p\%X=0.0} and {\tt p\%Y=0.0}.

On exit, the components {\tt p\%X} and {\tt p\%Y}
will contain the best estimates of the solution variables $\bmx$, and
Lagrange multipliers for the constraints $\bmy$.
\restrictions {\tt p\%n} $> 0$ and, {\tt p\%m} $\geq 0$,
and {\tt prob\%H\_type} and {\tt prob\%A\_type} 
 $\in \{${\tt 'DENSE'}, {\tt 'COORDINATE'}, 
{\tt 'SPARSE\_BY\_ROWS'}, {\tt 'DIAGONAL'} $\}$. 

\itt{data} is a scalar \intentinout\ argument of type 
{\tt \packagename\_data\_type}
(see Section~\ref{typedata}). It is used to hold data about the problem being 
solved. It must not have been altered {\bf by the user} since the last call to 
{\tt \packagename\_initialize}.

\itt{control} is a scalar \intentinout\ argument of type 
{\tt \packagename\_control\_type}
(see Section~\ref{typecontrol}). Default values may be assigned by calling 
{\tt \packagename\_initialize} prior to the first call to 
{\tt \packagename\_solve}.

\itt{inform} is a scalar \intentout\ argument of type 
{\tt \packagename\_inform\_type}
(see Section~\ref{typeinform}). A successful call to
{\tt \packagename\_solve}
is indicated when the  component {\tt status} has the value 0. 
For other return values of {\tt status}, see Section~\ref{galerrors}.

\end{description}

%%%%%%%%% resolve subroutine %%%%%%

\subsubsection{The resolve subroutine}
Once {\tt \packagename\_solve} has been called, further 
quadratic programs, for which the data $\bmg$, $\bmc$ and $f$ may have 
been altered but $\bmA$ and $\bmH$ are unchanged, may be solved 
more efficiently as follows:
\vspace*{1mm}

\hspace{8mm}
{\tt CALL \packagename\_resolve( p, data, control, inform )}

\vspace*{-3mm}
\begin{description}
\ittf{p} is a scalar \intentinout\ argument of type 
{\tt QPT\_problem\_type} as described for {\tt \packagename\_solve}
but for which only the components {\tt \%G} , {\tt \%f} and {\tt \%C} 
may have been altered since the last call to 
{\tt \packagename\_solve}.
As before, on exit, the components {\tt p\%X} and {\tt p\%Y}
will contain the best estimates of the solution variables $\bmx$, and
Lagrange multipliers for the constraints $\bmy$.

\itt{data}, {\tt control} and {\tt inform} are precisely as described for
 {\tt \packagename\_solve}.
\end{description}

%%%%%%% termination subroutine %%%%%%

\subsubsection{The  termination subroutine}
All previously allocated arrays are deallocated as follows:
\vspace*{1mm}

\hspace{8mm}
{\tt CALL \packagename\_terminate( data, control, info )}

%\vspace*{-3mm}
\begin{description}

\itt{data} is a scalar \intentinout\ argument of type 
{\tt \packagename\_data\_type} 
exactly as for
{\tt \packagename\_solve},
which must not have been altered {\bf by the user} since the last call to 
{\tt \packagename\_initialize}.
On exit, array components will have been deallocated.

\itt{control} is a scalar \intentin\ argument of type 
{\tt \packagename\_control\_type}
exactly as for
{\tt \packagename\_solve}.

\itt{inform} is a scalar \intentout\ argument of type
{\tt \packagename\_inform\_type}
exactly as for
{\tt \packagename\_solve}.
Only the component {\tt status} will be set on exit, and a 
successful call to 
{\tt \packagename\_terminate}
is indicated when this  component {\tt status} has the value 0. 
For other return values of {\tt status}, see Section~\ref{galerrors}.

\end{description}

%%%%%%%%%%%%%%%%%%%%%% Warning and error messages %%%%%%%%%%%%%%%%%%%%%%%%

\galerrors
A negative value of {\tt inform\%status} on exit from 
{\tt \packagename\_solve}
or 
{\tt \packagename\_terminate}
indicates that an error has occurred. No further calls should be made
until the error has been corrected. Possible values are:

\begin{description}

\itt{\galerrallocate.} An allocation error occurred. 
A message indicating the offending 
array is written on unit {\tt control\%error}, and the returned allocation 
status and a string containing the name of the offending array
are held in {\tt inform\%alloc\_\-status}
and {\tt inform\%bad\_alloc} respectively.

\itt{\galerrdeallocate.} A deallocation error occurred. 
A message indicating the offending 
array is written on unit {\tt control\%error} and the returned allocation 
status and a string containing the name of the offending array
are held in {\tt inform\%alloc\_\-status}
and {\tt inform\%bad\_alloc} respectively.

\itt{\galerrrestrictions.} One of the restrictions 
{\tt prob\%n} $> 0$ or {\tt prob\%m} $\geq  0$
    or requirements that {\tt prob\%A\_type} 
    and {\tt prob\%H\_type} contain its relevant string
    {\tt 'DENSE'}, {\tt 'COORDINATE'}, {\tt 'SPARSE\_BY\_ROWS'}
    or {\tt 'DIAGONAL'}
    has been violated.

\itt{\galerrprimalinfeasible.} The constraints appear to be inconsistent.

\itt{\galerranalysis.} An error was reported by the subroutine 
{\tt SILS\_analyse} called by {\tt SBLS}. 
The return status from {\tt SILS\_analyse} is given in one of the
components of {\tt inform\%SBLS\_inform\%SLS\_inform}.
See the documentation for the \galahad\ package {\tt SILS} for further details.

\itt{\galerrfactorization.} An error was reported by the subroutine 
{\tt SILS\_factorize}called by {\tt SBLS}. 
The return status from {\tt SILS\_factorize} is given in one of the
components of {\tt inform\%SBLS\_inform\%SLS\_inform}.
See the documentation for the \galahad\ package {\tt SILS} for further details.

\itt{\galerrsolve.} An error was reported by the subroutine 
{\tt SILS\_solve} called by {\tt SBLS}.  
The return  status from {\tt SILS\_solve} is given in one of the
components of {\tt inform\%SBLS\_inform\%SLS\_inform}.
See the documentation for the \galahad\ package {\tt SILS} for further details.

\itt{\galerrulsanalysis.} An error was reported by the subroutine 
{\tt ULS\_analyse} called by {\tt SBLS}. 
The return status from {\tt ULS\_analyse} is given in one of the
components of {\tt inform\%SBLS\_inform\%ULS\_inform}.
See the documentation for the \galahad\ package {\tt ULS} for further details.

\itt{\galerrulssolve.} An error was reported by the subroutine 
{\tt ULS\_solve} called by {\tt SBLS}. 
The return status from {\tt ULS\_solve} is given in one of the
components of {\tt inform\%SBLS\_inform\%ULS\_inform}.
See the documentation for the \galahad\ package {\tt ULS} for further details.

\itt{\galerrpreconditioner.} The computed preconditioner has the 
wrong inertia and is thus unsuitable.

\itt{\galerrillconditioned.} The residuals from the preconditioning step 
are large, indicating that the factorization may be unsatisfactory.

\itt{\galerrinput.} {\tt \packagename\_resolve} has been called
before {\tt \packagename\_solve}.

\end{description}

%%%%%%%%%%%%%%%%%%%%%% Further features %%%%%%%%%%%%%%%%%%%%%%%%

\galfeatures
\noindent In this section, we describe an alternative means of setting 
control parameters, that is components of the variable {\tt control} of type
{\tt \packagename\_control\_type}
(see Section~\ref{typecontrol}), 
by reading an appropriate data specification file using the
subroutine {\tt \packagename\_read\_specfile}. This facility
is useful as it allows a user to change  {\tt \packagename} control parameters 
without editing and recompiling programs that call {\tt \packagename}.

A specification file, or specfile, is a data file containing a number of 
"specification commands". Each command occurs on a separate line, 
and comprises a "keyword", 
which is a string (in a close-to-natural language) used to identify a 
control parameter, and 
an (optional) "value", which defines the value to be assigned to the given
control parameter. All keywords and values are case insensitive, 
keywords may be preceded by one or more blanks but
values must not contain blanks, and
each value must be separated from its keyword by at least one blank.
Values must not contain more than 30 characters, and 
each line of the specfile is limited to 80 characters,
including the blanks separating keyword and value.



The portion of the specification file used by 
{\tt \packagename\_read\_specfile}
must start
with a "{\tt BEGIN \packagename}" command and end with an 
"{\tt END}" command.  The syntax of the specfile is thus defined as follows:
\begin{verbatim}
  ( .. lines ignored by EQP_read_specfile .. )
    BEGIN EQP
       keyword    value
       .......    .....
       keyword    value
    END 
  ( .. lines ignored by EQP_read_specfile .. )
\end{verbatim}
where keyword and value are two strings separated by (at least) one blank.
The ``{\tt BEGIN \packagename}'' and ``{\tt END}'' delimiter command lines 
may contain additional (trailing) strings so long as such strings are 
separated by one or more blanks, so that lines such as
\begin{verbatim}
    BEGIN EQP SPECIFICATION
\end{verbatim}
and
\begin{verbatim}
    END EQP SPECIFICATION
\end{verbatim}
are acceptable. Furthermore, 
between the
``{\tt BEGIN \packagename}'' and ``{\tt END}'' delimiters,
specification commands may occur in any order.  Blank lines and
lines whose first non-blank character is {\tt !} or {\tt *} are ignored. 
The content 
of a line after a {\tt !} or {\tt *} character is also 
ignored (as is the {\tt !} or {\tt *}
character itself). This provides an easy manner to "comment out" some 
specification commands, or to comment specific values 
of certain control parameters.  

The value of a control parameters may be of three different types, namely
integer, logical or real.
Integer and real values may be expressed in any relevant Fortran integer and
floating-point formats (respectively). Permitted values for logical
parameters are "{\tt ON}", "{\tt TRUE}", "{\tt .TRUE.}", "{\tt T}", 
"{\tt YES}", "{\tt Y}", or "{\tt OFF}", "{\tt NO}",
"{\tt N}", "{\tt FALSE}", "{\tt .FALSE.}" and "{\tt F}". 
Empty values are also allowed for 
logical control parameters, and are interpreted as "{\tt TRUE}".  

The specification file must be open for 
input when {\tt \packagename\_read\_specfile}
is called, and the associated device number 
passed to the routine in device (see below). 
Note that the corresponding 
file is {\tt REWIND}ed, which makes it possible to combine the specifications 
for more than one program/routine.  For the same reason, the file is not
closed by {\tt \packagename\_read\_specfile}.

Control parameters corresponding to the components 
{\tt SBLS\_control}
and
{\tt GLTR\_control} may be changed by including additional sections enclosed by
``{\tt BEGIN SBLS}'' and 
``{\tt END SBLS}'', and
``{\tt BEGIN GLTR}'' and 
``{\tt END GLTR}'', respectively. 
See the specification sheets for the packages 
{\tt \libraryname\_SBLS} 
and
{\tt \libraryname\_GLTR}
for further details.

\subsubsection{To read control parameters from a specification file}
\label{readspec}

Control parameters may be read from a file as follows:
\hskip0.5in 

\def\baselinestretch{0.8}
{\tt 
\begin{verbatim}
     CALL EQP_read_specfile( control, device )
\end{verbatim}
}
\def\baselinestretch{1.0}

\begin{description}
\itt{control} is a scalar \intentinout argument of type 
{\tt \packagename\_control\_type}
(see Section~\ref{typecontrol}). 
Default values should have already been set, perhaps by calling 
{\tt \packagename\_initialize}.
On exit, individual components of {\tt control} may have been changed
according to the commands found in the specfile. Specfile commands and 
the component (see Section~\ref{typecontrol}) of {\tt control} 
that each affects are given in Table~\ref{specfile}.

\bctable{|l|l|l|} 
\hline
  command & component of {\tt control} & value type \\ 
\hline
  {\tt error-printout-device} & {\tt \%error} & integer \\
  {\tt printout-device} & {\tt \%out} & integer \\
  {\tt print-level} & {\tt \%print\_level} & integer \\
%  {\tt preconditioner-used} & {\tt \%preconditioner} & integer \\
%  {\tt semi-bandwidth-for-band-preconditioner} & {\tt \%semi\_bandwidth} & integer \\
%  {\tt factorization-used} & {\tt \%factorization} & integer \\
%  {\tt maximum-column-nonzeros-in-schur-complement} & {\tt \%max\_col} & integer \\
%  {\tt initial-workspace-for-unsymmetric-solver} & {\tt \%len\_ulsmin} & integer \\
%  {\tt initial-integer-workspace}  & {\tt \%indmin} & integer \\
%  {\tt initial-real-workspace}  & {\tt \%valmin} & integer \\ 
%  {\tt maximum-refinements} & {\tt \%itref\_max} & integer \\
  {\tt maximum-number-of-cg-iterations} & {\tt \%cg\_maxit} & integer \\
  {\tt trust-region-radius} & {\tt \%radius} & real \\
  {\tt max-relative-infeasibility-allowed} & {\tt \%max\_infeasibility\_relative} & real \\
  {\tt max-absolute-infeasibility-allowed} & {\tt \%max\_infeasibility\_absolute} & real \\
%  {\tt pivot-tolerance-used}  & {\tt \%pivot\_tol} & real \\
%  {\tt pivot-tolerance-used-for-basis}  & {\tt \%pivot\_tol\_for\_basis} & real \\
%  {\tt zero-pivot-tolerance}  & {\tt \%zero\_pivot} & real \\
%  {\tt inner-iteration-fraction-optimality-required} & {\tt \%inner\_fraction\_opt} & real \\
  {\tt inner-iteration-relative-accuracy-required} & {\tt \%inner\_stop\_relative} & real \\
  {\tt inner-iteration-absolute-accuracy-required} & {\tt \%inner\_stop\_absolute} & real \\
%  {\tt find-basis-by-transpose}  & {\tt \%find\_basis\_by\_transpose} & logical \\
  {\tt remove-linear-dependencies}  & {\tt \%remove\_dependencies} & logical \\
  {\tt space-critical}   & {\tt \%space\_critical} & logical \\
  {\tt deallocate-error-fatal}   & {\tt \%deallocate\_error\_fatal} & logical \\
\hline

\ectable{\label{specfile}Specfile commands and associated 
components of {\tt control}.}

\itt{device} is a scalar \intentin argument of type default \integer,
that must be set to the unit number on which the specfile
has been opened. If {\tt device} is not open, {\tt control} will
not be altered and execution will continue, but an error message
will be printed on unit {\tt control\%error}.

\end{description}

%%%%%%%%%%%%%%%%%%%%%% Information printed %%%%%%%%%%%%%%%%%%%%%%%%

\galinfo
If {\tt control\%print\_level} is positive, information about the progress 
of the algorithm will be printed on unit {\tt control\-\%out}.
If {\tt control\%print\_level} $= 1$, the norm of the
constraint violation and the value of the objective function 
for both the feasibility and optimality phases are reported.
Additionally, if {\tt control\%print\_level} $= 2$, 
{\tt print\_level = 1} output from both {\tt SBLS} and {\tt GLTR}
occurs, summarising the factorization and iteration phases, as
well as timing statistics from the two phases.
If {\tt control\%print\_level} $\geq 3$ detailed output from {\tt SBLS} 
and {\tt GLTR} occurs which is unlikely to be useful to general users.

%%%%%%%%%%%%%%%%%%%%%% GENERAL INFORMATION %%%%%%%%%%%%%%%%%%%%%%%%

\galgeneral

\galcommon None.
\galworkspace Provided automatically by the module.
\galroutines None. 
\galmodules {\tt \packagename\_solve} calls the \galahad\ packages
{\tt GALAHAD\_\-CLOCK},
{\tt GALAHAD\_SY\-M\-BOLS}, \sloppy
{\tt GALAHAD\-\_\-SPACE}, 
{\tt GALAHAD\_QPT},
{\tt GALAHAD\_FDC},
{\tt GALAHAD\_SBPS}, 
{\tt GALAHAD\_GLTR} and
{\tt GALAHAD\_SPECFILE}.
\galio Output is under control of the arguments
 {\tt control\%error}, {\tt control\%out} and {\tt control\%print\_level}.
\galrestrictions {\tt prob\%n} $> 0$, {\tt prob\%m} $\geq  0$, 
{\tt prob\%A\_type} and {\tt prob\%H\_type} $\in \{${\tt 'DENSE'}, 
 {\tt 'COORDINATE'}, {\tt 'SPARSE\_BY\_ROWS'}, {\tt 'DIAGONAL'} $\}$. 
\galportability ISO Fortran~95 + TR 15581 or Fortran~2003. 
The package is thread-safe.

%%%%%%%%%%%%%%%%%%%%%% METHOD %%%%%%%%%%%%%%%%%%%%%%%%

\galmethod
Any finite solution $\bmx$ to the problem necessarily satisfies 
the primal optimality conditions
\eqn{pr}{\bmA \bmx + \bmc = \bmzero}
and the dual optimality conditions
\eqn{du}{
 \bmH \bmx + \bmg - \bmA^{T} \bmy = \bmzero,}
where the components of the vector $\bmy$ are 
known as the Lagrange multipliers for the constraints.

A solution to the problem is found in two phases.
In the first, a point $\bmx_F$ satisfying \req{pr} is found.
In the second, the required solution $\bmx = \bmx_F + \bms$
is determined by finding $\bms$ to minimize 
$q(\bms) = \half \bms^T \bmH \bms + \bmg_F^T \bms + f_F^{}$
subject to the homogeneous constraints $\bmA \bms = \bmzero$,
where $\bmg_F^{} = \bmH \bmx_F^{} + \bmg$ and 
$f_F^{} = \half \bmx_F^T \bmH \bmx_F^{} + \bmg^T \bmx_F^{} + f$.
The required constrained minimizer of $q(\bms)$ is obtained
by implictly applying the preconditioned conjugate-gradient method
in the null space of $\bmA$. Any preconditioner of the form
\disp{ \bmK_{G} = \mat{cc}{ \bmG & \bmA^T \\ \bmA  & 0 }}
is suitable, and the \galahad\ package {\tt SBLS}
provides a number of possibilities. In order to ensure that the
minimizer obtained is finite, an additional, precautionary trust-region
constraint $\|s\| \leq \Delta$ for some suitable positive radius 
$\Delta$ is imposed, and the \galahad\ package {\tt GLTR} is used to solve 
this additionally-constrained problem.
\vspace*{1mm}

\galreferences
\vspace*{1mm}

\noindent
The preconditioning aspcets are described in detail in
\vspace*{1mm}

\noindent
H. S. Dollar, N. I. M. Gould and A. J. Wathen.
``On implicit-factorization constraint preconditioners''.
In  Large Scale Nonlinear Optimization (G. Di Pillo and M. Roma, eds.)
Springer Series on Nonconvex Optimization and Its Applications, Vol. 83,
Springer Verlag (2006) 61--82

\noindent
and

\noindent
H. S. Dollar, N. I. M. Gould, W. H. A. Schilders and A. J. Wathen
``On iterative methods and implicit-factorization preconditioners for 
regularized saddle-point systems''.
SIAM Journal on Matrix Analysis and Applications, {\bf 28(1)} (2006) 170--189,

\noindent
while the constrained conjugate-gradient method is discussed in
\vspace*{1mm}

\noindent  
N. I. M. Gould, S. Lucidi, M. Roma and Ph. L. Toint, 
Solving the trust-region subproblem using the Lanczos method. 
SIAM Journal on Optimization {\bf 9:2 } (1999), 504-525. 

%%%%%%%%%%%%%%%%%%%%%% EXAMPLE %%%%%%%%%%%%%%%%%%%%%%%%

\galexample
Suppose we wish to minimize
$\half x_1^2 + x_2^2 + \threehalves x_3^2 + 4 x_1 x_3 + 2 x_2 + 1$
subject to the the general linear constraints
$2 x_{1}  +  x_{2} - 2 = 0$ and
$x_{2} +  x_{3}  - 2 = 0$.
Then, on writing the data for this problem as
\disp{\bmH = \mat{ccc}{1 & & 4 \\ & 2 & \\ 4 &  & 3}, \;\;
 \bmg = \vect{ 0 \\ 2 \\ 0 }, \;\;
 \bmA = \mat{ccc}{ 2 & 1 & \\ & 1 & 1}
  \tim{and}
 \bmc = \vect{ -2 \\ -2 }}
in sparse co-ordinate format,
we may use the following code:

{\tt \small
\VerbatimInput{\packageexample}
}
\noindent
This produces the following output:
{\tt \small
\VerbatimInput{\packageresults}
}
\noindent
The same problem may be solved holding the data in 
a sparse row-wise storage format by replacing the lines
{\tt \small
\begin{verbatim}
!  sparse co-ordinate storage format
...
! problem data complete   
\end{verbatim}
}
\noindent
by
{\tt \small
\begin{verbatim}
! sparse row-wise storage format
   CALL SMT_put( p%H%type, 'SPARSE_BY_ROWS', s )  ! Specify sparse-by-row
   CALL SMT_put( p%A%type, 'SPARSE_BY_ROWS', s )  ! storage for H and A
   ALLOCATE( p%H%val( h_ne ), p%H%col( h_ne ), p%H%ptr( n + 1 ) )
   ALLOCATE( p%A%val( a_ne ), p%A%col( a_ne ), p%A%ptr( m + 1 ) )
   p%H%val = (/ 1.0_wp, 2.0_wp, 3.0_wp, 4.0_wp /) ! Hessian H
   p%H%col = (/ 1, 2, 3, 1 /)                     ! NB lower triangular
   p%H%ptr = (/ 1, 2, 3, 5 /)                     ! Set row pointers
   p%A%val = (/ 2.0_wp, 1.0_wp, 1.0_wp, 1.0_wp /) ! Jacobian A
   p%A%col = (/ 1, 2, 2, 3 /)
   p%A%ptr = (/ 1, 3, 5 /)                        ! Set row pointers  
! problem data complete   
\end{verbatim}
}
\noindent
or using a dense storage format with the replacement lines
{\tt \small
\begin{verbatim}
! dense storage format
   CALL SMT_put( p%H%type, 'DENSE', s )  ! Specify dense
   CALL SMT_put( p%A%type, 'DENSE', s )  ! storage for H and A
   ALLOCATE( p%H%val( n * ( n + 1 ) / 2 ) )
   ALLOCATE( p%A%val( n * m ) )
   p%H%val = (/ 1.0_wp, 0.0_wp, 2.0_wp, 4.0_wp, 0.0_wp, 3.0_wp /) ! Hessian
   p%A%val = (/ 2.0_wp, 1.0_wp, 0.0_wp, 0.0_wp, 1.0_wp, 1.0_wp /) ! Jacobian
! problem data complete   
\end{verbatim}
}
\noindent
respectively.

If instead $\bmH$ had been the diagonal matrix
\disp{\bmH = \mat{ccc}{1 & &   \\ & 0 & \\  &  & 3}}
but the other data is as before, the diagonal storage scheme 
might be used for $\bmH$, and in this case we would instead 
{\tt \small
\begin{verbatim}
   CALL SMT_put( prob%H%type, 'DIAGONAL', s )  ! Specify dense storage for H
   ALLOCATE( p%H%val( n ) )
   p%H%val = (/ 1.0_wp, 0.0_wp, 3.0_wp /) ! Hessian values
\end{verbatim}
}
\noindent
Notice here that zero diagonal entries are stored.

\end{document}

