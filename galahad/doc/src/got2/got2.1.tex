%
%   GALAHAD
%   started        17 March 2002
%   this version   12 April 2002
%
\documentclass[twoside]{article}
\usepackage{psfig,repdefs,ral,equations}
%\usepackage{psfig,repdefs,ral,equations,times,latexsym}

\newcommand{\fwbox}[1]{\fbox{\rule[-0.1cm]{0cm}{0.35cm}#1}}
\newcommand{\fwpbox}[1]{\fbox{\parbox{.44in}{#1}}}
\newcommand{\fwpbbox}[1]{\fbox{\parbox{.95in}{#1}}}

\textwidth  15.2cm 
\textheight 21.5cm
\oddsidemargin 0.0mm                      % -4.2 mm
\evensidemargin 0.0cm
\topmargin -8.4mm                          % 2.64 cm

\renewcommand{\argminover}[1]{ {\renewcommand{\arraystretch}{0.8}
                     \begin{array}[t]{c} 
                     \mbox{arg min} \vspace*{-1mm} \\ \mbox{$\scriptstyle #1 $} 
                     \end{array} } }
\newcommand{\gal}{{\sf GALAHAD}}
\newcommand{\lan}{{\sf LANCELOT}}
\newcommand{\lana}{{\sf LANCELOT A}}
\newcommand{\lanb}{{\sf LANCELOT B}}
\newcommand{\cute}{{\sf CUTE}}
\newcommand{\cuter}{{\sf CUTEr}}
\newcommand{\sifdec}{{\sf SifDec}}
\newcommand{\ltsubsection}[1]{\subsection{{\tt #1}} \label{#1}}
\newcommand{\ltsubsubsection}[1]{\subsubsection{{\tt #1}} \label{#1}}
\renewcommand{\itt}[1]{\item{{\tt #1}}}


\newcommand{\paperauthor}{Nicholas I. M. Gould, Dominique Orban 
and Philippe L. Toint}
\newcommand{\papertitle}{\gal, a library of thread-safe Fortran 90 
packages for large-scale nonlinear optimization}
\newcommand{\shortpapertitle}{\gal, a library 
for large-scale optimization}

\pagestyle{myheadings}
%\markboth{N. I. M. Gould, D. Orban \& Ph. L. Toint: 
%\shortpapertitle}{Draft version ({\tt got2.tex}) - not for circulation}
\markboth{\paperauthor}{\papertitle}

\title{\papertitle}
\author{\paperauthor}

\newcommand{\theabstract}{In this paper, we describe the
design of version 1.0 of \gal, a library of Fortran 90 packages for 
large-scale large-scale nonlinear optimization.
The library particularly addresses quadratic programming problems,
containing both interior point and active set variants, as well as
tools for preprocessing such problems prior to solution.
It also contains an updated version of the venerable nonlinear
programming package, \lan.}

\begin{document}

%\input{ralheader}

% ====================== start of header ==================

\begin{titlepage}

\begin{flushright} {\large \bf RAL-TR-2002-014} \end{flushright}           
\vspace*{0.2 cm}

{\LARGE \bf
\begin{center}
\gal, a library of thread-safe Fortran 90 \\
packages for large-scale nonlinear optimization
\end{center}}
\vspace*{0.1 cm}
\begin{center}
\mbox{} \\
      Nicholas I. M. Gould$^{1,2}$
      Dominique Orban$^{3}$
      and
      Philippe L. Toint$^{4,5}$
\\
\end{center}

\vspace{0.4cm}

\begin{center}
\parbox{\textwidth}{
{
{\bf  ABSTRACT \newline}
\theabstract}
}
\end{center}

\vspace{0.2 cm}

\noindent \rule{\textwidth}{0.001in}
\vspace{0.1 cm}

{\small
\begin{description}
\item  $^1$ Computational Science and Engineering Department,
       Rutherford Appleton Laboratory, 
       Chilton, \\ Oxfordshire, OX11 0QX, England, EU. 
       Email : n.gould@rl.ac.uk
\item  $^2$ Current reports available from %\\
       ``http://www.numerical.rl.ac.uk/reports/reports.html''.
\item $^3$ ECE Department,  Northwestern University, Evanston Il 60208, USA. 
       Email: orban@ece.northwestern.edu
\item  $^4$ Department of Mathematics, University of Namur, 
       61, rue de Bruxelles, B-5000 Namur, Belgium, EU. \\
       Email : philippe.toint@fundp.ac.be
\item  $^5$ Current reports available from %\\
       ``ftp://thales.math.fundp.ac.be/pub/reports''.
\end{description}
}

\vspace{0.6 cm}

\noindent 
Computational Science and Engineering Department
\\
Atlas Centre
\\
Rutherford Appleton Laboratory
\\
Oxfordshire OX11 0QX

\vspace{0.1 cm}
\noindent \today .
%\vspace{0.5 cm}

%\noindent \rule{\textwidth}{0.001in}

%\vspace{0.2 cm}
%\noindent
%Categories and subject descriptors: G.4 (Mathematical Software). \\
%General terms: Algorithm design and analysis. \\
%Keywords and phrases: \gal, nonlinear optimization, quadratic programming 
%large-scale, 
%Fortran 90.

\end{titlepage}

% Put in a table of contents

%\pagenumbering{roman}
%\tableofcontents
%\newpage
%\pagenumbering{arabic}

% ====================== end of header ==================

\setcounter{page}{1}
\lsection{Introduction}

We released our large-scale nonlinear programming package \lan\
(see \bciteb{ConnGoulToin92}) late in 1991. Over the intervening
years, \lana\ has been used by a large number of people, both
via free source downloads from our WWW sites
and by means of the NEOS facility (see \bciteb{CzyzMessMore98})
from the Optimization Technology Center at
Argonne National Laboratory and Northwestern University in the USA.
It is fair to say that we (and others) recognised its limitations
right from the outset, and it has long been our goal eventually to provide 
a suitable successor.

\lana\ was written in Fortran 77. 
Despite the widespread availability of good 
Fortran 77 compilers, the limitations of the language, particularly
the absence of standardised memory allocation facilities, proved to be
a serious limitation, especially for large problems. 
Work started in 1995 to re-implement \lana\ in
Fortran 90, particularly to take advantage of the new language's 
memory manipulation features and array constructs, and by 1997 we had a 
working prototype of the improved \lanb. We had chosen to stay with
Fortran rather than, say, C partially because many of the package's 
key (external) components, most especially the \citebb{hsl:2002}
sparse matrix codes
produced by our colleagues, are all Fortran based, and also because
we believed (and still believe) Fortran 90 capable of providing
all of the facilities we needed.

Regrettably, at or around that time, a number of our colleagues 
had started to release the results of comparative tests of their
new codes (for instance SNOPT, see \bciteb{gillmurrsaun02:siopt},
LOQO, see \bciteb{VandShan99}, 
KNITRO, see \bciteb{ByrdHribNoce99:siopt}, and
FilterSQP, see \bciteb{FletLeyf02}) against \lana,
and the results made frankly rather depressing reading for us
(see \bciteb{DolaMore00}, \bciteb{BensShanVand01}, and \bciteb{Chin01}), \lan\
often (but far from always) being significantly out performed. 
Quite clearly, the promise always held for large-scale 
sequential quadratic programming (SQP) methods,
based on how they outperformed other techniques in comparative tests
such as those due to \cite{HockSchi81}, was now being realised, and the 
limit of what might be achieved by augmented Lagrangian methods
such as \lana\ had likely been reached.

Reluctantly, we abandoned any plans to release \lanb\ at that time, and turned
our attention instead to SQP methods. To our minds, there had never
really been much doubt that SQP methods would be more successful in the
long term, but there had been  general concerns over 
how to (approximately) solve their all-important (large-scale) quadratic 
programming (QP) subproblems. Thus, we decided that our next goal
should be to produce high-quality QP codes for eventual incorporation 
in our own SQP algorithm(s). Since we believe that there might
be considerable interest from others in such codes, we have decided to release
these before we have finalised our SQP solver(s). And since we realised that
far from producing a single package, we are now in effect building a 
library of independent but inter-related packages, we have chosen to
release an (evolving) large-scale nonlinear optimization library, \gal.

In some sense \gal\ Version 1.0 is a stop-gap, since, although
it includes the upgraded B version of \lan, we doubt seriously whether 
\lanb\ is a state-of-the-art solver for general nonlinear programming
problems. What \gal\ V1.0 does provide are the quadratic programming solvers
and related tools that we anticipate will allow us to develop the 
next-generation SQP solvers 
which we intend to introduce in Version 2 of the library.

%In this paper, we describe the scope and content of \gal.

\lsection{Library contents}

Each package in \gal\ is written as a Fortran 90 module, and the codes
are all threadsafe. The default
precision for real arguments is double, but this is easily transformed
to single in a UNIX environment using provided {\tt sed} scripts. 
Each package has accompanying
documentation and a test program. The latter attempts to execute as 
much of the package as
is realistically possible---the fact that some of the packages are
intended for nonlinear problems makes it difficult to ensure that
every statement is executed, since some segments of code are
intended to cope with pathological behaviour that cannot be ruled out
in theory but which nevertheless seem never to occur in practice.
Options to packages may be passed both directly, through subroutine arguments,
and indirectly, via option-specification files. The second mechanism is
particularly useful when there is a hierarchy of packages for which
a user wishes to change an option for one of the dependent packages without
recompilation.

\lsubsection{Overview}

\gal\ comprises the following major packages:

\begin{description}

\itt{LANCELOT B} is a sequential augmented Lagrangian method for 
minimizing a (nonlinear) objective subject to general (nonlinear) constraints
and simple bounds.

\itt{QPB} is a primal-dual interior-point trust-region method for 
minimizing a general quadratic objective function over a polyhedral region.

\itt{QPA} is a active/working-set method for 
minimizing a general quadratic objective function over a polyhedral region.

\itt{LSQP} is an interior-point method for minimizing a linear or 
separable convex quadratic function over a polyhedral region.

\itt{PRESOLVE} is a method for preprocessing linear and quadratic programming 
problems prior to solution by other packages.

\itt{GLTR} is a method for minimizing a general quadratic objective function 
over the interior or boundary of a (scaled) hyper-sphere.

\itt{SILS} provides an interface to the HSL sparse-matrix package {\tt MA27} 
that is functionally equivalent to the more recent HSL package {\tt HSL\_MA57}.

\itt{SCU} uses a Schur complement update to find the solution of 
a sequence of linear systems for which the solution
involving a leading sub-matrix may be found by other means.

\end{description}
In addition, \gal\ contains the following auxiliary packages:

\begin{description}

\itt{QPP} reorders linear and quadratic programming 
problems to a convenient form prior to solution by other packages.

\itt{QPT} provides a derived type for holding linear and quadratic programming 
problems.

\itt{SMT} provides a derived type for holding sparse matrices in a variety of
formats.

\itt{SORT} gives implementations of both Quick- and Heap-sort methods

\itt{RAND} provides pseudo-random numbers.

\itt{SPECFILE} allows users to provide options to other packages, using lists
of keyword-value pairs given in package-dependent option-specification files.

\itt{SYMBOLS} assigns values to a list of commonly-used \gal\ variables.

\end{description}

\subsection{Details of major \gal\ packages\label{major}}

\subsubsection{{\tt LANCELOT B}\label{lanb}}

The enlivened \lanb\ offers a number of improvements
over its predecessor, but alas is still far from the state-of-the-art.
\lana\ is fully described in \citebb{ConnGoulToin92}, and the results
of comprehensive tests are given in \citebb{ConnGoulToin96a}.
New features include
\begin{itemize}
\item the automatic allocation of workspace,
\item a non-monotone descent strategy (see \bciteb{Toin96b}, and 
\bciteb{ConnGoulToin00}, \S10.1) to be used by default,
\item the optional use of \citebb{MoreTora91}-type projections 
   (see also \bciteb{LinMore99b}) 
   during the subproblem solution phase,
\item an interface to \citebbs{LinMore99} public domain incomplete Cholesky 
   factorization package ICFS for use as a preconditioner, and
\item the optional use of structured trust regions to better model 
   structured problems (see \bciteb{ConnGoulSartToin96a}, and
   \bciteb{ConnGoulToin00}, \S10.2), and
\item more flexibility over the choice of derivatives---derivatives need
   only be provided for a subset of the element functions from which the 
   problem is built, the remainder being estimated by differences or secant 
   approximations.  
\end{itemize}
The main reason for extending \lan's life is as a prototype
for what may be achieved using Fortran 90/95 in preparation for 
future \gal\ SQP solvers, since the problem data structure and
interface is unlikely to change significantly.

To illustrate the effects of the new features, 
both \lanb\ (using a number of these options) and \lana\
were run on all the examples (except linear and quadratic programs) from
the \cuter\ (see \bciteb{GoulOrbaToin02b})
test set---the \cuter\ set differs from its predecessor
in \cute\ (see \bciteb{BongConnGoulToin95})
both in the number of problems, and by virtue of a long-overdue
increase in default dimensions for all variable-dimensional problems.  
The options new to \lanb\ that we considered were as follows:
\begin{itemize}
%\begin{enumerate}
\item The default (in which a non-monotone descent strategy, with a
history length of 1, a band preconditioner, with semi-bandwidth 5, exact second
derivatives ) is used.
\item The default, except that a monotone descent strategy is used.
\item The default, except that SR1 approximations to the second derivatives
are used.
\item The default, except that the \citebbs{LinMore99} incomplete Cholesky 
   factorization preconditioner, ICFS, with 5 extra work vectors, is used.
\item the default, except that the \citebb{MoreTora91} projected search,
with 5 restarts, is used.
\item The default, except that a structured trust region
   (see \bciteb{ConnGoulSartToin96a}), is used.
\item The default, except that the history length for the 
non-monotone descent strategy is increased to 5, is used.
%\end{enumerate}
\end{itemize}
\lana\ was run with its defaults, since an exhaustive test
of other \lana\ options has already been performed 
(see \bciteb{ConnGoulToin96a}).

In Figure~\ref{prof.ps}
the performance profiles (see \bciteb{DolaMore01}) for the CPU time
(in seconds) and the numbers of function evaluations  
from these tests are reported. All of our experiments were performed 
on a single processor of a Compaq AlphaServer DS20 with 3.5 Gbytes of RAM,
using the Compaq f90 compiler with full, machine specific, optimization,
and each run was terminated unsuccessfully after 30 minutes or after 10000
function evaluations.
Of the 749 problems thus considered (and for the best of the 
options), 151 (roughly 20\%) 
were terminated due to excessive CPU times/function calls or because
the evaluation of problem functions lead to floating-point exceptions---of
those requiring more than 30 minutes/10000 calls, most could ultimately 
be solved by increasing the CPU and iteration limits.
\begin{figure}[htbp]
\centerline{\psfig{figure=time.ps,height=10.0cm,silent=}
}\centerline{
            \psfig{figure=its.ps,height=10.0cm,silent=}}
\caption{Performance profile for LANCELOT options: CPU times (top)
and number of function evaluations (bottom). The horizontal axis is
gives the argument $\sigma$, while the vertical axis records $p_i(\sigma)$ 
for each of the competing options, $i$.}
\label{prof.ps}
\end{figure}

Suppose that a given algorithm $i$ from a competing set $\calA$
reports a statistic $s_{ij} \geq 0$ 
when run on example $j$ from our test set $\calT$,
and that the smaller this statistic the better the algorithm is considered.
Let 
\disp{k(s,s^*,\sigma) = 
 \left\{ 
\begin{array}{rl}
 1 & \mbox{if} \;\; s \leq \sigma s^* \\ 0 & \mbox{otherwise.}
\end{array} 
\right.
}
Then, the {\em performance profile} of algorithm $i$ is the function 
\disp{p_i(\sigma) = \frac{ 
\sum_{j \in \calT} k(s_{i,j},s^*_{j},\sigma)}{|\calT|} \;\; \sigma \geq 1,}
where $s^*_j = \min_{i \in \calA} s_{ij}$.
Thus $p_i(1)$ gives the fraction of the number of examples for which 
algorithm $i$ was the most effective (according to statistics $s_{ij}$),
$p_i(2)$ gives 
the fraction of the number for which algorithm $i$ 
is within a factor of 2 of the best, and $\lim_{\sigma\longrightarrow\infty}
p_i(\sigma)$ gives the fraction of the examples for which the algorithm
succeeded. We consider such a profile to be a very effective means
of comparing the relative merits of the new options available in \lanb.

The benefits of the non-monotone and Lin-Mor\'{e} strategies are
apparent both in terms of CPU time and function evaluation reductions.
In addition, we are pleased to see that 
the best of the new options show some gain with respect to 
\lana, particularly as we were initially concerned that moving from 
Fortran 77 to 90 might give rise to some performance penalties.
The only new option that we are disappointed with is the use of a
structured trust region, and, on the basis of these tests, 
we cannot really recommend this strategy. The full set of results
are available as a internal report (see \bciteb{goulorbatoin02c}).

\ltsubsubsection{QPB}

The module {\tt QPB} is an implementation of a feasible-interior-point
primal-dual trust-region method for quadratic programming.

To set the scene, the {\em quadratic programming} problem is to
\eqn{1.2}{\minin{x \in \smallRe^n} q(x) \equiv 
\half x^T H x + g^T x}
subject to the general linear constraints
\eqn{1.3}{c_{i}^{l}  \leq  a_{i}^{T}x  \leq  c_{i}^{u}, \;\;\; 
 i = 1, \ldots , m,}
and the simple bound constraints
\eqn{1.4}{x_{j}^{l}  \leq  x_{j}^{ } \leq  x_{j}^{u} , \;\;\; 
 j = 1, \ldots , n.}
for given
vectors $g$, $a_{i}$, $c^{l}$, $c^{u}$, $x^{l}$, $x^{u}$ 
and a given symmetric (but not necessarily definite) matrix $H$---equality 
constraints and fixed variables are allowed by setting
$c_{i}^{u} = c_{i}^{l}$ and $x_{j}^{u} = x_{j}^{l}$ as required,
and any or all of the constraint bounds may be infinite.
The required solution $x$ to \req{1.2}--\req{1.4} necessarily satisfies 
the primal optimality conditions
\eqn{4.1a}{A x = c}
and 
\eqn{4.1b}{
 c^{l} \leq c \leq c^{u}, \;\;
x^{l} \leq x \leq x^{u},}
the dual optimality conditions
\eqn{4.2a}{
 H x + g =
 A^{T} y + z,\;\;
 y = y^{l} + y^{u} \tim{and}
 z = z^{l} + z^{u} ,}
and 
\eqn{4.2b}{
 y^{l} \geq 0 , \;\;
 y^{u} \leq 0 , \;\;
 z^{l} \geq 0 \;\; \mbox{and} \;\;
 z^{u} \leq 0 ,}
and the complementary slackness conditions 
\eqn{4.3}{
( A x - c^{l} )^{T} y^{l} = 0  ,\;\;
( A x - c^{u} )^{T} y^{u} = 0  ,\;\;
(x -x^{l} )^{T} z^{l} = 0   \tim{and}
(x -x^{u} )^{T} z^{u} = 0 ,}
where the vectors $y$ and $z$ are 
known as the Lagrange multipliers for
the general linear constraints, and the dual variables for the bounds,
respectively, and where the vector inequalities hold componentwise.

Primal-dual interior point methods iterate towards a point
that satisfies the optimality conditions \req{4.1a}--\req{4.3}
by ultimately aiming to satisfy
\req{4.1a}, \req{4.2a} and \req{4.3}, while ensuring that 
\req{4.1b} and \req{4.2b} are 
satisfied as strict inequalities at each stage. 
Appropriate norms of the amounts by 
which \req{4.1a}, \req{4.2a} and \req{4.3} fail to be satisfied are known as the
primal and dual infeasibility, and the violation of complementary slackness,
respectively. The fact that \req{4.1b} and \req{4.2b} are satisfied as strict 
inequalities gives such methods their name, interior-point methods.

The problem is solved in two phases. The goal of the first 
``initial feasible point'' phase is
to find a strictly interior point which is primal feasible, that is that
\req{4.1a} is satisfied. The \gal\ package {\tt LSQP} (see \S\ref{LSQP})
is used for this purpose, and offers the options of either accepting the first 
strictly feasible point found, or preferably of aiming for the
so-called ``analytic center'' of the feasible region.
Having found such a suitable initial feasible point, the second ``optimality''
phase ensures that \req{4.1a} remains satisfied while iterating to
satisfy dual feasibility \req{4.2a} and complementary slackness \req{4.3}.
The optimality phase proceeds by approximately minimizing a 
sequence of barrier functions
\disp{q(x) - \mu \left[ \sum_{i=1}^{m} \log ( c_{i}  -  c_{i}^{l} )
 + \sum_{i=1}^{m} \log ( c_{i}^{u}  -  c_{i} )
 + \sum_{j=1}^{n} \log ( x_{j}  -  x_{j}^{l} ) 
 + \sum_{j=1}^{n} \log ( x_{j}^{u}  -  x_{j} ) \right] ,}
for an appropriate sequence of positive barrier parameters $\mu$ 
converging to zero
while ensuring that \req{4.1a} remains satisfied and that 
$x$ and $c$ are strictly interior points for \req{4.1b}. 
Note that terms in the above summations corresponding to infinite bounds are
ignored, and that equality constraints are treated specially.

Each of the barrier subproblems is solved using a trust-region method.
Such a method generates a trial correction step $(\delta x, \delta c)$
to the current iterate $(x, c)$
by replacing the nonlinear barrier function locally by a suitable 
quadratic model, and approximately minimizing this model in the 
intersection of \req{4.1a}
and a trust region $\|( \delta x,  \delta c)\| \leq \Delta$
for some appropriate
strictly positive trust-region radius $\Delta$ and norm $\| \cdot \|$.
The step is accepted/rejected
and the radius adjusted on the basis of how accurately the model reproduces the
value of barrier function at the trial step. If the step proves to be 
unacceptable, a linesearch is performed along the step to obtain an acceptable 
new iterate. In practice, the natural primal ``Newton'' model of the barrier 
function is almost always 
less successful than an alternative primal-dual model, 
and consequently the primal-dual model is usually to be preferred.

The trust-region subproblem is approximately solved using the 
combined conjugate-gradient/Lan\-czos method (see \bciteb{GoulHribNoce98} and
\bciteb{GoulLuciRomaToin99:siopt}) implemented in the \gal\ module {\tt GLTR}
(see \S\ref{GLTR}).
Such a method requires a suitable preconditioner,
and in our case, the only flexibility we have is in approximating the
model of the Hessian. Although using a fixed form of preconditioning is
sometimes effective, we have provided the option of an automatic choice,
that aims to balance the cost of applying the preconditioner against
the needs for an accurate solution of the trust-region subproblem. 
The preconditioner is applied using the \gal\ factorization package
{\tt SILS} (see \S\ref{SILS})---or optionally using {\tt HSL\_MA57} 
from HSL if this is available---but options at this stage are
to factorize the preconditioner as a whole (the so-called ``augmented system''
approach), or to perform a block elimination first (the ``Schur-complement''
approach). The latter is usually to be preferred when a (non-singular) diagonal 
preconditioner is used, but may be inefficient if any of the columns
of $A$ is too dense.

The theoretical justification of the
overall scheme, for problems with general objectives and 
inequality constraints, 
is given by \citebb{ConnGoulOrbaToin00:mp}, in which we also present
numerical results that suggest that it is indeed able to solve 
some problems of the size we had been aiming for. 
More recently, we investigated the ultimate rate of convergence
of such schemes, and have shown that, under fairly general conditions, 
a componentwise superlinear rate is achievable both for quadratic and general 
nonlinear programs (see \bciteb{GoulOrbaSartToin01}).

Full advantage is taken of any zero coefficients in the matrix $H$ or the
vectors $a_{i}$.
An older version of {\tt QPB} (using {\tt HSL\_MA57} rather than {\tt SILS},
see \S\ref{SILS}, but without some of the most recent features in {\tt QPB})
is available commercially as {\tt HSL\_VE12} within HSL.

\ltsubsubsection{QPA}

The module {\tt QPA} is an implementation of a second approach
to quadratic programming, this time of the active/working set variety.
Although general in
scope, {\tt QPA} is intended within \gal\ to deal with the case 
where a good
estimate of the optimal active set has been determined (and thus that
relatively few iterations will be required). The method is actually more
general in scope, and is actually geared toward solving
{\em $\mathbf{\ell_1}$-quadratic programming} problems of the form
\eqn{1.1}{\minin{x \in \smallRe^n} q(x) + \rho_g v_g(x) +
\rho_b v_b(x)}
involving the quadratic objective $q(x)$ and the infeasibilities
\disp{v_g(x) = 
   \sum_{i=1}^{m} \max ( c_{i}^{l} - a_i^T x, 0 )
 + \sum_{i=1}^{m} \max ( a_i^T x - c_{i}^{u}, 0 )}
and
\disp{v_b(x) = 
   \sum_{j=1}^{n} \max ( x_{j}^{l} - x_{j}  , 0 )
 + \sum_{j=1}^{n} \max ( x_{j}  - x_{j}^{u} , 0 ) .}

At the $k$-th iteration of the method, an improvement to the value
of the merit function 
$m(x, \rho_g, \rho_b )$ $= q(x) + \rho_g v_g(x) + \rho_b v_b(x)$
at $x = x^{(k)}$ is sought. This is achieved by first 
computing a search direction $s^{(k)}$,
and then setting $x^{(k+1)} = x^{(k)} + \alpha^{(k)} s^{(k)}$,
where the stepsize $\alpha^{(k)}$ is chosen as the first local minimizer of 
$\phi ( \alpha ) = m( x^{(k)} + \alpha s^{(k)} , \rho_g, \rho_b )$
as $\alpha$ increases from zero. 
The stepsize calculation is straightforward, and exploits the fact that
$\phi ( \alpha )$ is a piecewise quadratic function of $\alpha$.

The search direction is defined by a subset of the ``active'' terms in 
$v(x)$, i.e., those for which 
$a_i^T x = c_i^l$ or $c_i^u$ (for $i=1,\ldots ,m$) or 
$x_j = x_j^l$ or $x_j^u$ (for {j=1,\ldots ,n}).
The ``working'' set $W^{(k)}$ is chosen from the active terms, and is such 
that its members have linearly independent gradients. 
The search direction $s^{(k)}$ is chosen as an approximate solution of 
the equality-constrained quadratic program
\eqn{4.1}{\minin{s \in \smallRe^n} q(x^{(k)} + s) + 
 \rho_g l_g^{(k)} (s) + \rho_b l_b^{(k)} (s),}
subject to 
\eqn{4.2}{a_i^T s = 0,\;\;  i \in \{ 1, \ldots , m \} \cap W^{(k)},
\tim{and}
x_j = 0, \;\;  i  \in \{1, \ldots , n \} \cap W^{(k)},}
where
\disp{l_g^{(k)} (s) = 
     - \sum_{\stackrel{i=1}{a_i^T x < c_i^l}}^m a_i^T s 
    \; + \sum_{\stackrel{i=1}{a_i^T x > c_i^u}}^m a_i^T s}
and
\disp{l_b^{(k)} (s) = 
    - \sum_{\stackrel{j=1}{x_j < x_j^l}}^n s_j
    \; + \sum_{\stackrel{j=1}{x_j > x_j^u}}^n s_j .}
The equality-constrained quadratic program \req{4.1}--\req{4.2} is solved by
a projected preconditioned conjugate gradient 
method (see \bciteb{GoulHribNoce98}). The method terminates
either after a pre-specified number of iterations, or if the solution is found,
or if a direction of infinite descent, along which 
$q(x^{(k)} + s) + \rho_g l_g^{(k)} (s) + \rho_b l_b^{(k)} (s)$
decreases without bound within the feasible region \req{4.2}, is located. 
Successively more accurate approximations are required as suspected 
solutions of \req{1.1} are approached.

Preconditioning of the conjugate gradient iteration
requires the solution of one or more linear systems of the form
\eqn{4.3a}{\mat{cc}{M^{(k)} & A^{(k)T} \\ A^{(k)} & 0 }
\vect{ p \\ u} = \vect{ g \\ 0 },}
where $M^{(k)}$ is a ``suitable'' approximation to $H$
and the rows of $A^{(k)}$ comprise the gradients of the
terms in the current working set. Rather than recomputing a
factorization of the preconditioner at every iteration, a
Schur complement method is used, recognising the fact that
gradual changes occur to successive working sets. The main
iteration is divided into a sequence of ``major'' iterations.
At the start of each major iteration (say, the overall iteration $l$), 
a factorization of the
current ``reference'' matrix, that is the matrix
\eqn{4.4}{\mat{cc}{M^{(l)} & A^{(l)T} \\ A^{(l)} & 0 }}
is obtained using  the \gal\ factorization package
{\tt SILS} (see \S\ref{SILS})---or, once again, 
optionally using {\tt HSL\_MA57} from HSL if this is available.
This reference matrix may be factorized as a whole, using the
augmented system approach, or by performing a block elimination first, using 
the ``Schur-complement'' approach. The latter is usually to be preferred 
when a (non-singular) diagonal 
preconditioner is used, but may be inefficient if any of the columns
of $A^{(l)}$ is too dense.
Subsequent iterations within the current major
iteration obtain solutions to \req{4.3a} via the factors of \req{4.4}
and an appropriate (dense) Schur complement,
obtained from {\tt SCU} in \gal\ (see \S\ref{SCU}).
The major iteration terminates
once the space required to hold the factors of the (growing) Schur
complement exceeds a given threshold.

The working set changes by (a) adding an active term encountered during 
the determination of the stepsize, or (b) the removal of a term if $s = 0$
solves \req{4.1}--\req{4.2}. The  decision on which to remove in the latter 
case is based upon the expected decrease upon the removal of an individual term,
and this information is available from the magnitude and sign of the components
of the auxiliary vector $u$ computed in \req{4.3a}. At optimality, the
components of $u$ for $a_i$ terms will all lie between 
$0$ and $\rho_g$---and those for the other terms 
between $0$ and $\rho_b$---and any violation
of this rule indicates further progress is possible. 
%The components
%of $u$ corresponding to the terms involving $a_i^T x$
%are sometimes known as Lagrange multipliers (or generalized gradients) and
%denoted by $y$, while those for the remaining $x_j$ terms are dual variables
%and denoted by $z$.

To solve quadratic programs of the form \req{1.2}--\req{1.4}, 
a sequence of problems of the form \req{1.1} are
solved, each with a larger value of $\rho_g$ and/or $\rho_b$ 
than its predecessor. The
required solution has been found once the infeasibilities 
$v_g(x)$ and $v_b(x)$ have been reduced to zero at the solution of 
\req{1.1} for the given $\rho_g$ and $\rho_b$.

Having proposed and implemented two very different quadratic programming
methods, an obvious question is: how do the methods compare. We examined
this question in \citebb{GoulToin01c} by comparing {\tt QPA} and {\tt QPB}
on the \cuter\ (see \bciteb{GoulOrbaToin02b}) QP test set.

While for modest sized problems, started from ``random'' points, the two
methods are roughly comparable, the advantages of the interior-point approach
become quite clear when problem dimensions increase. For problems involving
tens of thousands of unknowns and/or constraints, our active set approach
simply takes too many iterations, while the number of iterations 
required by the interior point approach seems relatively insensitive to 
dimension size. For general problems involving hundreds of thousands or
even millions of unknowns/constraints, the active set approach is impractical,
while we illustrate in Table~\ref{figa} that
{\tt QPB} is able to solve problems of this size.
{\tt QPB} also appears to scale well with dimension, as can be seen 
see in Table~\ref{figb}.

\begin{table}[ht]
%\begin{table}[h]
\begin{center}
\begin{tabular}[c]{|lrrc|rr|}
\hline
%          &      &  &    & \multicolumn{2}{c|}{QPB} \\
Name      & $n$ & $m$ & type & its & time\, \\
%\hline
%{\tt GOULDQP2}  & 200001 & 100000 & C &  17  & {52} \\
%{\tt GOULDQP3}  & 200001 & 100000 & C &  46  & {154} \\
\hline
{\tt QPBAND}  & 100000 & 50000  & C &   13  & {157} \\
{\tt QPBAND}  & 200000 & 100000 & C &   17  & {1138} \\
{\tt QPBAND}  & 400000 & 200000 & C &   17  & {2304} \\
{\tt QPBAND}  & 500000 & 250000 & C &   17  & {2909} \\
\hline
{\tt QPNBAND}  & 100000 & 50000  & NC &   12  & {32} \\
{\tt QPNBAND}  & 200000 & 100000 & NC &   13  & {71} \\
{\tt QPNBAND}  & 400000 & 200000 & NC &   14  & {156} \\
{\tt QPNBAND}  & 500000 & 250000 & NC &   13  & {181} \\
\hline
\end{tabular}
\end{center}
\caption{\label{figa} {\tt QPB} on large QP examples.
Runs performed on a Compaq AlphaServer DS20 (3.5 Gbytes RAM),
time in CPU seconds. $n$ is the number of unknowns, and 
$m$ is the number of general constraints. 
C indicates a convex problem, while NC is a convex one.}
\end{table}

\begin{table}[htb]
%\begin{table}[b]
\begin{center}
\begin{tabular}[c]{|lrrc|rr|}
\hline
%          &      &  &    & \multicolumn{2}{c|}{QPB} \\
Name      & $n$ & $m$ & type & its & time\, \\
\hline
{\tt PORTSQP}  & 10 & 1  & C & 11    & {0.02} \\
{\tt PORTSQP}  & 100 & 1  & C &  15   & {0.03} \\
{\tt PORTSQP}  & 1000 & 1  & C &  26   & {0.09} \\
{\tt PORTSQP}  & 10000 & 1  & C &  37   & {1.26} \\
{\tt PORTSQP}  & 100000 & 1  & C &  20   & {9.48} \\
{\tt PORTSQP}  & 1000000 & 1  & C &  11   & {72.31} \\
\hline
{\tt PORTSNQP}  & 10 & 2  & NC & 21    & {0.03} \\
{\tt PORTSNQP}  & 100 & 2  & NC &  30   & {0.04} \\
{\tt PORTSNQP}  & 1000 & 2  & NC &  39   & {0.17} \\
{\tt PORTSNQP}  & 10000 & 2  & NC &  32   & {1.70} \\
{\tt PORTSNQP}  & 100000 & 2  & NC &  107   & { 58.69} \\
{\tt PORTSNQP}  & 1000000 & 2  & NC &  22   & {209.53} \\
\hline
\end{tabular}
\end{center}
\caption{\label{figb} How {\tt QPB} scales with dimension.
Notation as for Table~\ref{figa}.}
\end{table}

While such figures might seem to indicate that {\tt QPB} should always
be preferred to {\tt QPA}, this is not the case. In particular, if a
good estimate of the solution---more particularly, the optimal active set---
is known, active-set methods may exploit this, while interior-point methods
are (currently) less able to do so. In particular
\citebb{GoulToin01c} illustrate that {\tt QPA} often outperforms {\tt QPB}
on warm-started problems unless the problem is (close to) degenerate or
very ill conditioned. Thus, since nonlinear optimization (SQP) algorithms 
often solve a sequence of related problems for which the optimal active sets
are almost or actually identical, there is good reason to maintain both
{\tt QPA} and {\tt QPB} in \gal.

An enhanced version of {\tt QPA} (using {\tt HSL\_MA57} rather than {\tt SILS},
see \S\ref{SILS}) is available commercially as {\tt HSL\_VE19} within HSL.

\ltsubsubsection{LSQP}

{\tt LSQP} is an interior-point method for minimizing a linear or 
separable convex quadratic function 
\disp{\mbox{minimize}\;\; \half {\sum_{i=1}^n w_i^2 
 ( x_{i}^{ } - x_{i}^{0} )^{2}} + g^T x,}
for given weights $w$ and gradient $g$, 
over the polyhedral region \req{1.3}--\req{1.4}.
In the special case where $w = 0$ and $g = 0$
the so-called analytic center of the feasible set will be found,
while linear programming, or constrained least distance, problems
may be solved by picking $w = 0$, or $g = 0$, respectively.
The basic algorithm is that of \citebb{Zhan94}, and is quite standard.
{\tt LSQP} is used within \gal\ by {\tt QPB}.

Note that since predictor-corrector steps are not taken, the method
is unlikely to be as efficient as state-of-the-art interior-point 
methods for linear programming. We intend to remedy this defect for
Version 2.

\ltsubsubsection{PRESOLVE}

The module {\tt PRESOLVE} is intended to pre-process
quadratic programming problems. 
The purpose of preprocessing (or presolving) 
quadratic programming problems is to exploit the 
optimality equations \req{4.1a}--\req{4.3} so as to both simplify the problem
and reduce the problem to a standard form (that makes subsequent 
manipulation easier), defined as follows: 
\begin{itemize}
\item The variables are ordered so that their bounds appear in the order
\begin{center}
\begin{tabular}{lccccc}
free                &          &        & $x_j$ &        &          \\
non-negativity      &   0      & $\leq$ & $x_j$ &        &          \\
lower               & $x^l_j$ & $\leq$ & $x_j$ &        &          \\
range               & $x^l_j$ & $\leq$ & $x_j$ & $\leq$ & $x^u_j$ \\
upper               &          &        & $x_j$ & $\leq$ & $x^u_j$ \\
non-positivity      &          &        & $x_j$ & $\leq$ &      0   \\
\end{tabular}
\end{center}

Fixed variables are removed. Within each category, the variables 
are further ordered so that those with non-zero diagonal Hessian 
entries occur before the remainder.

\item
The constraints are ordered so that their bounds appear in the order
\begin{center}
\begin{tabular}{lccccc}
non-negativity      &     0    & $\leq$ & $( A x)_i$ &        &          \\
equality            & $c^l_i$ &   $=$  & $(A x)_i$ &        &          \\
lower               & $c^l_i$ & $\leq$ & $(A x)_i$ &        &          \\
range               & $c^l_i$ & $\leq$ & $(A x)_i$ & $\leq$ & $c^u_i$ \\
upper               &          &        & $(A x)_i$ & $\leq$ & $c^u_i$ \\
non-positivity      &          &        & $(A x)_i$ & $\leq$ &     0    \\
\end{tabular}
\end{center}
Free constraints are removed. 

\item
In addition, constraints may be removed or bounds tightened, to reduce the
size of the feasible region or simplify the problem if this is possible, and
bounds may be tightened on the dual variables and the multipliers 
associated  with the problem. 
\end{itemize}

The presolving algorithm (see \bciteb{GoulToin01}) 
proceeds by applying a (potentially long) series of
simple transformations to the problem, each transformation introducing a
further simplification of the problem. These involve the removal of empty and
singleton rows, the removal of redundant and forcing primal constraints, the
tightening of primal and dual bounds, the exploitation of linear singleton,
linear doubleton and linearly unconstrained columns, merging of dependent
variables, row ``sparsification'' and splitting equalities. Transformations are
applied in successive passes, each pass involving the following actions:
\begin{enumerate}
\item remove empty and singletons rows,
\item try to eliminate variables that are linearly unconstrained,
\item attempt to exploit the presence of linear singleton columns,
\item attempt to exploit the presence of linear doubleton columns,
\item complete the analysis of the dual constraints,
\item remove empty and singletons rows, 
\item possibly remove dependent variables,
\item analyze the primal constraints, 
\item try to make $A$ sparser by combining its rows,
\item check the current status of the variables, dual variables
      and multipliers.
\end{enumerate}
All these transformations are applied on the structure of the original
problem, which is only permuted to standard form after all transformations are
completed. The reduced problem may
then solved by a quadratic or linear programming solver. Finally, the
solution of the simplified problem is re-translated to the
variables/constraints format of the original problem in a ``restoration'' phase.

At the overall level, the presolving process follows one of the following two
sequences:

\vspace*{0.2cm}

\noindent
\begin{center}
\fwbox{initialize} 
$\rightarrow$ $\bigg[$ \fwbox{apply transformations}
$\rightarrow$ (solve problem) 
$\rightarrow$ \fwbox{restore}  $\bigg]$
$\rightarrow$ \fwbox{terminate}
\end{center}
\vspace*{0.2cm}

\noindent or

\vspace*{0.2cm}
\noindent
\begin{center}
\fwbox{initialize} 
$\rightarrow$ $\bigg[$ \fwpbox{read specfile} 
$\rightarrow$ \fwpbbox{apply\\transformations}
$\rightarrow$ $\left(\mbox{\parbox{.5in}{solve problem}}\right)$
$\rightarrow$ \fwbox{restore}  $\bigg]$
$\rightarrow$ \fwbox{terminate}
\end{center}
\vspace*{0.2cm}

\noindent
where the procedure's control parameter may be modified by reading an external
``specfile'', and where (solve problem) indicates
that the reduced problem is solved. Each of the ``boxed'' steps in these
sequences corresponds to calling a specific routine of the package, while
bracketed subsequence of steps means that they can be repeated with problems 
having the same structure. 

An implementation is available as {\tt PRESOLVE} in the \gal\ library. 
\citebb{GoulToin01} indicate that, when considering all 178 linear 
and quadratic programming problems in the \cute\ 
(see \bciteb{BongConnGoulToin95}) test set,
an average reduction of roughly 20\% in both the number of unknowns and
the number of constraints results from applying {\tt PRESOLVE}.
When applying the \gal\ QP solver {\tt QPB} (see \S\ref{QPB}), 
an overall average reduction 
of roughly 10\% in CPU time results. In some cases, the gain is dramatic.
For example, for the problems {\tt GMNCASE4}, {\tt STNPQ1}, {\tt STNQP2} 
and {\tt SOSQP1}, {\tt PRESOLVE}
removes all the variables and constraints, and thus reveals the complete 
solution to the problem without resorting to a QP solver.

Currently {\tt PRESOLVE} is not embedded within {\tt QPA}/{\tt B} or
{\tt LSQP} and must be called separately, but we intend to correct this
defect for Version 2.0.

\ltsubsubsection{GLTR}

{\tt GLTR} aims to find the global solution to the problem of minimizing
the quadratic function \req{1.2} where the variables are 
required to satisfy the constraint $\|x\|_{M} \leq  \Delta$, 
where the $M$-norm of $x$ is $\|x\|_{M} = \sqrt{x^T M x}$ for some symmetric, 
positive definite $M$, and where $\Delta > 0$ is a given scalar.
This problem commonly occurs as a trust-region subproblem in nonlinear 
optimization calculations, and is used within \gal\ by {\tt QPB}.
The method may be suitable for large $n$ as no factorization of $H$ is 
required. Reverse communication is used to obtain  
matrix-vector products of the form $H z$ and $M^{-1} z$. 
The package may also be used to solve the related problem in which $x$ is 
instead required to satisfy the 
{\bf equality constraint $\|x\|_{M} = \Delta$}.
The method is described in detail in 
\citebb{GoulLuciRomaToin99:siopt}, and {\tt GLTR} is a slightly improved 
version of the HSL package {\tt HSL\_VF05}.

\ltsubsubsection{SILS}

The module {\tt SILS} provides a Fortran 90 interface to the Fortran 77 HSL
sparse linear equation package {\tt MA27} \cite{DuffReid82}. 
The interface and functionality are 
designed to be identical to the more recent HSL Fortran 90 
package {\tt HSL\_MA57} \cite{Duff01}, enabling anyone with {\tt HSL\_MA57}
to easily substitute this for {\tt SILS} throughout \gal. The reason
that we are forced to rely on {\tt MA27} rather than the superior 
{\tt HSL\_MA57} by default is simply that
the former is available without charge from the HSL Archive
({\tt http://hsl.rl.ac.uk/hslarchive},) while the latter is only available
commercially.
{\tt SILS} (and hence wither {\tt MA27} or {\tt HSL\_MA57}) 
is required by {\tt QPA}/{\tt B} and {\tt LSQP}, and is
used optionally by {\tt LANCELOT B}.

\ltsubsubsection{SCU}

{\tt SCU} may be used to find the solution
to an extended system of $n + m$ 
sparse real linear equations in $n  +  m$ unknowns,
\disp{\mat{cc}{ A & B \\ C & D } \vect{x_1 \\ x_2}
 =  \vect{b_1 \\ b_2}.}
in the case where the $n$ by $n$ matrix $A$ is nonsingular 
and solutions to the systems 
\disp{A x  =  b \tim{and} A^T y  =  c}
may be obtained from an external source, such as an existing 
factorization.  The subroutine uses reverse communication to obtain 
the solution to such smaller systems.  The method makes use of 
the Schur complement matrix 
\disp{S  =  D  -  C    A^{-1} B.}
The Schur complement is stored and factorized as a dense matrix 
and the subroutine is thus appropriate only if there is 
sufficient storage for this matrix. Special advantage is taken 
of symmetry and definiteness in the coefficient matrices. 
Provision is made for introducing additional rows and columns 
to, and removing existing rows and columns from, the extended matrix. 

{\tt SCU} is used by both \lanb\ and {\tt QPA} to cope
with core linear systems that arise as constraints and variables
enter and leave their active/working sets. A slightly simplified
version of {\tt SCU} is available in HSL as {\tt HSL\_MA69}.

\lsection{Installation}

Just as for its immediate predecessors \cuter\ and \sifdec\
(see \bciteb{GoulOrbaToin02b}),
\gal\ is designed to be used in a multi-platform, multi-compiler 
environment in which core (source and script) files are available
in a single location, and machine/compiler/operating system specific
components (most especially compiled libraries of binaries) are isolated in
uniquely identifiable subdirectories. As before, we have concentrated on 
UNIX platforms, principally since we have no experience with other systems.

\gal\ is provided as a series of files and directories, all lying beneath
a root directory, which we shall refer to as {\tt \$GALAHAD}. The
directory structure is illustrated in Figure~\ref{dir_structure}.

\begin{figure}[htbp]
\begin{center}
%\setlength{\unitlength}{0.028in}
\setlength{\unitlength}{0.033in}
\begin{picture}(145,230)(10,-200)
\put(0,-100){\framebox(30,7){{\tt  \$GALAHAD}}}
\put(30,-96.5){\line(1,0){2.5}}
\put(32.5,-3){\line(0,-1){180}}
%
\put(65,-3){\line(1,0){2.5}}
\put(67.5,17){\line(0,-1){40}}
\put(67.5,17){\line(1,0){2.5}} \put(70,14){\framebox(30,6){{\tt lanb}}}
\put(67.5,7){\line(1,0){2.5}} \put(70,4){\framebox(30,6){{\tt qpb}}}
\put(67.5,-3){\line(1,0){2.5}} \put(72,-3.3){\ldots other packages \ldots} 
\put(67.5,-13){\line(1,0){2.5}} \put(70,-16){\framebox(30,6){{\tt dum}}}
\put(67.5,-23){\line(1,0){2.5}} \put(70,-26){\framebox(30,6){{\tt makedefs}}}
%
%
\put(32.5,-3){\line(1,0){2.5}} \put(35,-6){\framebox(30,6){{\tt src}}}
\put(32.5,-33){\line(1,0){2.5}} \put(35,-36){\framebox(30,6){{\tt makefiles}}}
\put(32.5,-43){\line(1,0){2.5}} \put(35,-46){\framebox(30,6){{\tt seds}}}
\put(32.5,-63){\line(1,0){2.5}} \put(35,-66){\framebox(30,6){{\tt objects}}}
\put(32.5,-103){\line(1,0){2.5}} \put(35,-106){\framebox(30,6){{\tt versions}}}
\put(32.5,-93){\line(1,0){2.5}} \put(35,-96){\framebox(30,6){{\tt arch}}}
\put(32.5,-123){\line(1,0){2.5}} \put(35,-126){\framebox(30,6){{\tt modules}}}

\put(32.5,-153){\line(1,0){2.5}} \put(35,-156){\framebox(30,6){{\tt bin}}}
\put(65,-153){\line(1,0){2.5}}
\put(67.5,-153){\line(1,0){2.5}} \put(70,-156){\framebox(30,6){{\tt sys}}}
\put(32.5,-163){\line(1,0){2.5}} \put(35,-166){\framebox(30,6){{\tt doc}}}
\put(65,-173){\line(1,0){2.5}}
\put(67.5,-173){\line(1,0){2.5}} \put(70,-176){\framebox(30,6){{\tt man1}}}
\put(32.5,-173){\line(1,0){2.5}} \put(35,-176){\framebox(30,6){{\tt man}}}
\put(32.5,-183){\line(1,0){2.5}} \put(35,-186){\framebox(30,6){{\tt examples}}}

\put(65,-63){\line(1,0){2.5}}
\put(67.5,-43){\line(0,-1){40}}
\put(67.5,-43){\line(1,0){2.5}} \put(70,-46){\framebox(34,6){{\tt architecture}}}
\put(67.5,-63){\line(1,0){2.5}} \put(70,-66){\framebox(34,6){{\tt architecture2}}}
\put(67.5,-83){\line(1,0){2.5}} \put(72,-83.3){\ldots other architectures \ldots} 

\put(104,-43){\line(1,0){2.5}}
\put(104,-63){\line(1,0){2.5}}
\put(106.5,-38){\line(0,-1){10}}
\put(106.5,-58){\line(0,-1){10}}

\put(106.5,-38){\line(1,0){2.5}} \put(109,-41){\framebox(30,6){{\tt double}}}
\put(106.5,-48){\line(1,0){2.5}} \put(109,-51){\framebox(30,6){{\tt single}}}
\put(106.5,-58){\line(1,0){2.5}} \put(109,-61){\framebox(30,6){{\tt double}}}
\put(106.5,-68){\line(1,0){2.5}} \put(109,-71){\framebox(30,6){{\tt single}}}


\put(65,-123){\line(1,0){2.5}}
\put(67.5,-103){\line(0,-1){40}}
\put(67.5,-103){\line(1,0){2.5}} \put(70,-106){\framebox(34,6){{\tt architecture}}}
\put(67.5,-123){\line(1,0){2.5}} \put(70,-126){\framebox(34,6){{\tt architecture2}}}
\put(67.5,-143){\line(1,0){2.5}} \put(72,-143.3){\ldots other architectures \ldots} 

\put(104,-103){\line(1,0){2.5}}
\put(104,-123){\line(1,0){2.5}}
\put(106.5,-98){\line(0,-1){10}}
\put(106.5,-118){\line(0,-1){10}}

\put(106.5,-98){\line(1,0){2.5}} \put(109,-101){\framebox(30,6){{\tt double}}}
\put(106.5,-108){\line(1,0){2.5}} \put(109,-111){\framebox(30,6){{\tt single}}}
\put(106.5,-118){\line(1,0){2.5}} \put(109,-121){\framebox(30,6){{\tt double}}}
\put(106.5,-128){\line(1,0){2.5}} \put(109,-131){\framebox(30,6){{\tt single}}}

\end{picture}
\caption{\label{dir_structure}Structure of the \gal\ directories}
\end{center}
\end{figure}

Before installation the sub-directories 
{\tt objects}, 
{\tt modules}, 
{\tt makefiles}, 
{\tt versions}
and 
{\tt bin/sys}
will all be empty. The installation script, {\tt install\_galahad}, prompts
the user for the answers to a series of questions aimed at determining
what machine type, operating system and compiler (from a large list of 
predefined possibilities encoded in the {\tt arch} sub-directory) 
to build \gal\ for---we 
call this combination of a machine, operating system and
compiler an \emph{architecture}. Each architecture is assigned a simple
(mnemonic) architecture {\em code} name, say {\tt architecture}---for example
a version for the NAG Fortran 95 compiler on a PC runing Linux is
coded {\tt pc.lnx.n95}, while another for the 
Compaq Fortran 90 compiler on an Alpha system running Tru64 Unix is
{\tt alp.t64.f90}.
Having determined the architecture, the installation script
builds sub-directories of {\tt objects} and {\tt modules} named 
{\tt architecture}, as well as further sub-directories 
{\tt double} and {\tt single} of these to hold architecture-dependent 
compiled libraries and module file information. In addition, 
architecture-dependent makefile information and environment variables for
execution scripts are placed in files named {\tt architecture} in the
{\tt makefiles} and {\tt bin/sys} sub-directories, and a file recording how
the code is related to the architecture is put in {\tt versions}.

The Fortran source codes for each \gal\ package occurs 
in a separate sub-directory of the {\tt src} directory---the sub-directory
contains the package source, a comprehensive test program (along with
a simpler second test program, which is used as an illustration on how
to call the package in the accompanying documentation), and a makefile.
Since the order of compilation of Fortran modules is important, and as
we have seen there is a strong interdependency between the \gal\ packages,
the makefiles have to be carefully crafted. For this reason (and unlike
for \cuter), we have chosen not to use tools such as {\tt imake} to build and
maintain the makefiles. Postscript and PDF Documentation for all packages
is contained in {\tt doc}.

Not every component of \gal\ is distributed as part of the package. In
particular, the HSL Archive code {\tt MA27} must be downloaded prior to 
the installation from
\begin{verbatim}
    http://hsl.rl.ac.uk/archive/hslarchive.html 
\end{verbatim}
before any of the QP packages can be used (it is optionally used by \lanb).
Additionally \gal\ makes optional use of the 
Lin-Mor\'{e} preconditioner {\tt ICFS}, available as
part of the MINPACK 2 library, via
\begin{verbatim}
    http://www-unix.mcs.anl.gov/~more/icfs/ 
\end{verbatim}
and the HSL codes {\tt MA57} and {\tt MC61} which are only available
commercially. If any non-default code is used, the file {\tt packages}
in the directory {\tt src/makedefs} must be edited to describe where
the external code may be found---details are given in {\tt packages}.
Default dummy versions of all optional codes are provided in 
{\tt src/dum} to ensure that linking prior to execution works properly.

Once the correct directory structure is in place, the installation script
builds a random library of the required precision by visiting each 
of the sub-directories of {\tt src} and calling the Unix utility {\tt make}.
\gal\ packages are all written in double precision, but if a user prefers to
use single precision, the makefiles call suitable Unix {\tt sed} 
scripts (stored in {\tt seds}) to transform the source prior to compilation.
A user may choose to install all of \gal, or just the QP or \lanb\ components.
There are also {\tt sed} features to automatically switch from {\tt SILS} (see
 \S\ref{SILS}) to {\tt MA57} if the latter is available.
The make option {\tt make tests} runs comprehensive tests of all 
compiled components.

\section{Interfaces to the \cuter\ test set}

As well as being stand-alone Fortran modules, we provide transparent 
interfaces between \lanb, {\tt QPA}/{\tt B}, {\tt LSQP} and {\tt PRESOLVE} and
problems written in the Standard Input Format (SIF, see
\bciteb{ConnGoulToin92}), most particularly
the \cuter\ (see \bciteb{GoulOrbaToin02b}) test set. To use these
interfaces \lanb\ users will need to have installed the package 
\sifdec, while users wishing to use the interfaces to the QP packages 
will additionally need \cuter\ (see \bciteb{GoulOrbaToin02b}). 

To run one of the supported packages on an example stored in, say, 
{\tt EXAMPLE.SIF}, a user needs to issue the command
\begin{verbatim}
    sdgal code package EXAMPLE.[SIF] 
\end{verbatim}
where {\tt code} is the architecture code discussed in \S\ref{Installation},
{\tt package} defines the package to be used---it may be one of {\tt lanb},
{\tt qpa}, {\tt qpb} or {\tt pre}, with access to {\tt LSQP} 
provided via {\tt qpb}---and the suffix [{\tt .SIF}] is optional.
This command translates the SIF file into Fortran subroutines and 
related data using the decoder provided in \sifdec, and then calls
the required optimization package to solve the problem. Once a problem
has been decoded, it may be re-called (perhaps with different options)
using the auxiliary command
\begin{verbatim}
    gal code package 
\end{verbatim}
A few SIF examples are given in the {\tt examples} sub-directory, while the
{\tt sdgal} and {\tt gal} commands are in the {\tt bin} sub-directory, and
have man-page descriptions in the {\tt man/man1} sub-directory.

One additional feature that has been added is that if the user has
access to the HSL automatic differentiation packages {\tt HSL\_AD01} 
or {\tt HSL\_AD02} (see \bciteb{PrycReid98}), 
these may be used to generate automatic first and second derivatives
for the element and group function (see \bciteb{ConnGoulToin92}) 
from which the overall problem is re-assembled by \cuter\ and \lanb.
Of course, ideally, it would be better to use one of the more 
commonly-occurring packages 
such as  ADIFOR (see \bciteb{BiscCarlCorlGrieHovl92}),
and we plan to do this in the future.

As with \lana, options may be passed directly to the solvers by means of
user-definable option-specification files. Each SIF interface 
has its own set of options, but overall control is maintained via 
the \gal\ module {\tt SPECFILE}.

%\newpage
\lsection{Availability}

\gal\ may be downloaded from its WWW site,
\begin{verbatim}
    http://galahad.rl.ac.uk/galahad-www
\end{verbatim}
There are restrictions on commercial use, and all users are required to
agree to a number of minor conditions of use.

\lsection{Conclusions}

We have described the scope and design of the first release of \gal,
a library of Fortran 90 packages for nonlinear optimization.
Version 1.0 of the library particularly addresses quadratic programming 
problems, although there is an updated version of \lan\ for more general 
problems. The package may be downloaded from its WWW site,
{\tt http://galahad.rl.ac.uk/galahad-www}.

In future, we intend to use the quadratic programming packages as the
basic tool within one or more SQP methods for nonlinear optimization.
We are currently developing AMPL (see \bciteb{FourGayKern93}) interfaces 
for the principal packages so that users will be able to use a more
natural interface than provided by SIF. In addition, we plan to incorporate
the preprocessing tools directly into the QP solvers, rather than having
them stand-alone as is currently the case.

\newpage

%\bibliographystyle{ralf}
%\input{/home/nimg/sp/refs/refs}
\begin{thebibliography}{xx}

\harvarditem[Benson et al.]{Benson, Shanno and Vanderbei}{2001}{BensShanVand01}
H.~Benson, D.~F. Shanno, and R.~J. Vanderbei.
\newblock A comparative study of large-scale nonlinear optimization algorithms.
\newblock Technical Report ORFE 01-04, Operations Research and Financial
  Engineering, Princeton University, New Jersey, USA, 2001.

\harvarditem[Bischof et al.]{Bischof, Carle, Corliss, Griewank and
  Hovland}{1992}{BiscCarlCorlGrieHovl92}
Chr. Bischof, A.~Carle, G.~Corliss, A.~Griewank, and P.~Hovland.
\newblock {ADIFOR} - generating derivative codes from fortran programs.
\newblock {\em Scientific Programming}, {\bf 1},~1--29, 1992.

\harvarditem[Bongartz et al.]{Bongartz, Conn, Gould and
  Toint}{1995}{BongConnGoulToin95}
I.~Bongartz, A.~R. Conn, N.~I.~M. Gould, and Ph.~L. Toint.
\newblock {\sf CUTE}: {C}onstrained and {U}nconstrained {T}esting
  {E}nvironment.
\newblock {\em ACM Transactions on Mathematical Software}, {\bf
  21}(1),~123--160, 1995.

\harvarditem[Byrd et al.]{Byrd, Hribar and Nocedal}{1999}{ByrdHribNoce99:siopt}
R.~H. Byrd, M.~E. Hribar, and J.~Nocedal.
\newblock An interior point algorithm for large scale nonlinear programming.
\newblock {\em SIAM Journal on Optimization}, {\bf 9}(4),~877--900, 1999.

\harvarditem[Chin]{Chin}{2001}{Chin01}
C.~M. Chin.
\newblock Numerical results of {SLPSQP}, filter{SQP} and {LANCELOT} on selected
  {CUTE} test problems.
\newblock Numerical Analysis Report NA/203, Department of Mathematics,
  University of Dundee, Scotland, 2001.

\harvarditem[Conn et al.]{Conn, Gould and Toint}{1992}{ConnGoulToin92}
A.~R. Conn, N.~I.~M. Gould, and Ph.~L. Toint.
\newblock {\em {\sf LANCELOT}: a {F}ortran package for Large-scale Nonlinear
  Optimization ({R}elease {A})}.
\newblock Springer Series in Computational Mathematics. Springer Verlag,
  Heidelberg, Berlin, New York, 1992.

\harvarditem[Conn et al.]{Conn, Gould and Toint}{1996{\em a}}{ConnGoulToin96a}
A.~R. Conn, N.~I.~M. Gould, and Ph.~L. Toint.
\newblock Numerical experiments with the {{\sf LANCELOT}} package ({R}elease
  {A}) for large-scale nonlinear optimization.
\newblock {\em Mathematical Programming, Series~A}, {\bf 73}(1),~73--110,
  1996{\em a}.

\harvarditem[Conn et al.]{Conn, Gould and Toint}{2000{\em a}}{ConnGoulToin00}
A.~R. Conn, N.~I.~M. Gould, and Ph.~L. Toint.
\newblock {\em Trust-region methods}.
\newblock SIAM, Philadelphia, 2000{\em a}.

\harvarditem[Conn et al.]{Conn, Gould, Orban and Toint}{2000{\em
  b}}{ConnGoulOrbaToin00:mp}
A.~R. Conn, N.~I.~M. Gould, D.~Orban, and Ph.~L. Toint.
\newblock A primal-dual trust-region algorithm for non-convex nonlinear
  programming.
\newblock {\em Mathematical Programming}, {\bf 87}(2),~215--249, 2000{\em b}.

\harvarditem[Conn et al.]{Conn, Gould, Sartenaer and Toint}{1996{\em
  b}}{ConnGoulSartToin96a}
A.~R. Conn, N.~I.~M. Gould, A.~Sartenaer, and Ph.~L. Toint.
\newblock Convergence properties of minimization algorithms for convex
  constraints using a structured trust region.
\newblock {\em SIAM Journal on Optimization}, {\bf 6}(4),~1059--1086, 1996{\em
  b}.

\harvarditem[Czyzyk et al.]{Czyzyk, Mesnier and Mor\'{e}}{1998}{CzyzMessMore98}
J. Czyzyk, M.~P. Mesnier, and J.~J. Mor\'{e}.
\newblock The {NEOS} {S}erver,.
\newblock {\em IEEE Journal on Computational Science and Engineering}, {\bf
  5},~68--75, 1998.

\harvarditem[Dolan and Mor\'{e}]{Dolan and Mor\'{e}}{2000}{DolaMore00}
E.~D. Dolan and J.~J. Mor\'{e}.
\newblock Benchmarking optimization software with cops.
\newblock Technical Report ANL/MCS-246, Argonne National Laboratory, Illinois,
  USA, 2000.

\harvarditem[Dolan and Mor\'{e}]{Dolan and Mor\'{e}}{2001}{DolaMore01}
E.~D. Dolan and J.~J. Mor\'{e}.
\newblock Benchmarking optimization software with performance profiles.
\newblock Technical Report ANL/MCS-P861-1200, Argonne National Laboratory,
  Illinois, USA, 2001.

\harvarditem[Duff]{Duff}{2001}{Duff01}
I.~S. Duff.
\newblock {MA57} - a new code for the solution of sparse symmetric indefinite
  systems.
\newblock Technical Report (in preparation), Rutherford Appleton Laboratory,
  Chilton, Oxfordshire, England, 2001.

\harvarditem[Duff and Reid]{Duff and Reid}{1982}{DuffReid82}
I.~S. Duff and J.~K. Reid.
\newblock {MA}27: A set of {F}ortran subroutines for solving sparse symmetric
  sets of linear equations.
\newblock Report R-10533, {AERE} {H}arwell Laboratory, Harwell, UK, 1982.

\harvarditem[Fletcher and Leyffer]{Fletcher and Leyffer}{2002}{FletLeyf02}
R.~Fletcher and S.~Leyffer.
\newblock Nonlinear programming without a penalty function.
\newblock {\em Mathematical Programming}, {\bf 91}(2),~239--269, 2002.

\harvarditem[Fourer et al.]{Fourer, Gay and Kernighan}{1993}{FourGayKern93}
R.~Fourer, D.~M. Gay, and B.~W. Kernighan.
\newblock {\em {AMPL}: A modelling language for mathematical programming}.
\newblock The Scientific Press, San Fransisco, California, USA, 1993.

\harvarditem[Gill et al.]{Gill, Murray and
  Saunders}{2002}{gillmurrsaun02:siopt}
P.~E. Gill, W.~Murray, and M.~A. Saunders.
\newblock {SNOPT}: An {SQP} algorithm for large-scale constrained optimization.
\newblock {\em SIAM Journal on Optimization}, {\bf (to appear)}, 2002.

\harvarditem[Gould and Toint]{Gould and Toint}{2001}{GoulToin01c}
N.~I.~M. Gould and Ph.~L. Toint.
\newblock Numerical methods for large-scale non-convex quadratic programming.
\newblock Technical Report RAL-TR-2001-017, Rutherford Appleton Laboratory,
  Chilton, Oxfordshire, England, 2001.

\harvarditem[Gould and Toint]{Gould and Toint}{2002}{GoulToin01}
N.~I.~M. Gould and Ph.~L. Toint.
\newblock Preprocessing for quadratic programming.
\newblock Technical Report RAL-TR-2002-001, Rutherford Appleton Laboratory,
  Chilton, Oxfordshire, England, 2002.

\harvarditem[Gould et al.]{Gould, Hribar and Nocedal}{2001}{GoulHribNoce98}
N.~I.~M. Gould, M.~E. Hribar, and J.~Nocedal.
\newblock On the solution of equality constrained quadratic problems arising in
  optimization.
\newblock {\em SIAM Journal on Scientific Computing}, {\bf 23}(4),~1375--1394,
  2001.

\harvarditem[Gould et al.]{Gould, Lucidi, Roma and Toint}{1999{\em
  a}}{GoulLuciRomaToin99:siopt}
N.~I.~M. Gould, S.~Lucidi, M.~Roma, and Ph.~L. Toint.
\newblock Solving the trust-region subproblem using the {L}anczos method.
\newblock {\em SIAM Journal on Optimization}, {\bf 9}(2),~504--525, 1999{\em
  a}.

\harvarditem[Gould et al.]{Gould, Orban and Toint}{2002{\em
  a}}{GoulOrbaToin02b}
N.~I.~M. Gould, D.~Orban, and Ph.~L. Toint.
\newblock {\sf CUTEr} (and \sf {SifDec}\rm), a constrained and unconstrained
  testing environment, revisited.
\newblock Technical Report RAL-TR-2002-009, Rutherford Appleton Laboratory,
  Chilton, Oxfordshire, England, 2002{\em a}.

\harvarditem[Gould et al.]{Gould, Orban and Toint}{2002{\em
  b}}{goulorbatoin02c}
N.~I.~M. Gould, D.~Orban, and Ph.~L. Toint.
\newblock Results from a numerical evaluation of \sf {LANCELOT B} \rm.
\newblock Numerical Analysis Group Internal Report 2002-1, Rutherford Appleton
  Laboratory, Chilton, Oxfordshire, England, 2002{\em b}.

\harvarditem[Gould et al.]{Gould, Orban, Sartenaer and Toint}{1999{\em
  b}}{GoulOrbaSartToin01}
N.~I.~M. Gould, D.~Orban, A.~Sartenaer, and Ph.~L. Toint.
\newblock Superlinear convergence of primal-dual interior-point algorithms for
  nonlinear programming.
\newblock {\em SIAM Journal on Optimization}, {\bf 11}(4),~974--1002, 1999{\em
  b}.

\harvarditem[Hock and Schittkowski]{Hock and Schittkowski}{1981}{HockSchi81}
W.~Hock and K.~Schittkowski.
\newblock {\em Test Examples for Nonlinear Programming Codes}.
\newblock Number 187 {\em In} `Lecture Notes in Economics and Mathematical
  Systems'. Springer Verlag, Heidelberg, Berlin, New York, 1981.

\harvarditem[HSL]{HSL}{2002}{hsl:2002}
HSL.
\newblock A collection of {F}ortran codes for large-scale scientific
  computation, 2002.
\newblock See http://www.cse.clrc.ac.uk/Activity/HSL.

\harvarditem[Lin and Mor\'{e}]{Lin and Mor\'{e}}{1999{\em a}}{LinMore99b}
C.~Lin and J.~J. Mor\'{e}.
\newblock Incomplete {C}holesky factorizations with limited memory.
\newblock {\em SIAM Journal on Scientific Computing}, {\bf 21}(1),~24--45,
  1999{\em a}.

\harvarditem[Lin and Mor\'{e}]{Lin and Mor\'{e}}{1999{\em b}}{LinMore99}
C.~Lin and J.~J. Mor\'{e}.
\newblock {N}ewton's method for large bound-constrained optimization problems.
\newblock {\em SIAM Journal on Optimization}, {\bf 9}(4),~1100--1127, 1999{\em
  b}.

\harvarditem[Mor\'{e} and Toraldo]{Mor\'{e} and Toraldo}{1991}{MoreTora91}
J.~J. Mor\'{e} and G.~Toraldo.
\newblock On the solution of large quadratic programming problems with bound
  constraints.
\newblock {\em SIAM Journal on Optimization}, {\bf 1}(1),~93--113, 1991.

\harvarditem[Pryce and Reid]{Pryce and Reid}{1998}{PrycReid98}
J.~D. Pryce and J.~K. Reid.
\newblock {\tt AD01}, a {F}ortran 90 code for automatic differentiation.
\newblock Technical Report RAL-TR-1998-057, Rutherford Appleton Laboratory,
  Chilton, Oxfordshire, England, 1998.

\harvarditem[Toint]{Toint}{1997}{Toin96b}
Ph.~L. Toint.
\newblock A non-monotone trust-region algorithm for nonlinear optimization
  subject to convex constraints.
\newblock {\em Mathematical Programming}, {\bf 77}(1),~69--94, 1997.

\harvarditem[Vanderbei and Shanno]{Vanderbei and Shanno}{1999}{VandShan99}
R.~J. Vanderbei and D.~F. Shanno.
\newblock An interior point algorithm for nonconvex nonlinear programming.
\newblock {\em Computational Optimization and Applications}, {\bf
  13},~231--252, 1999.

\harvarditem[Zhang]{Zhang}{1994}{Zhan94}
Y.~Zhang.
\newblock On the convergence of infeasible interior-point methods for the
  horizontal linear complementarity problem.
\newblock {\em SIAM Journal on Optimization}, {\bf 4}(1),~208--227, 1994.

\end{thebibliography}

\end{document}


